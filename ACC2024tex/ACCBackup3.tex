\documentclass{article}
\usepackage{times}
\usepackage{amsmath}
\usepackage{amssymb}
\usepackage{amsthm}
\usepackage{appendix}
\usepackage{graphicx}
\usepackage{xr}
\usepackage{verbatim}
\externaldocument[aux]{Auxillary.tex}
\usepackage{subfigure}

% \newtheorem{lemma}{Lemma}
% \newtheorem{problem}{Problem}
% \newtheorem{theorem}{Theorem}
% \newtheorem{assumption}{Assumption}
% \newtheorem{remark}{Remark}


% \newcommand{\contTilde}[1]{\mathbf{\tilde{#1}}}
% \newcommand{\contBar}[1]{\bar{\mathbf{#1}}}
% \newcommand{\transpose}{\mathsf{T}}
% \newcommand{\infoat}[1]{\mathcal{H}_{#1}}
% \newcommand{\opcontrol}[2]{\pi^{*}(#1,\infoat{#2})}
% \newcommand{\lqr}[3]{{#1}^{\transpose}Q_{#3}{#1}+{#2}^{\transpose}R_{#3}{#2}}
% \newcommand{\quadx}[2]{{#1}^{\transpose}{#2}{#1}}
% \newcommand{\gameVars}[2]{(x_{t})_{t=1}^{#1}, (u_{i,t})_{i=1,t=1}^{#2,#1}}
% \newcommand{\gameVarAtTime}[2]{x_{t},({#1}_{i,t})_{i=1}^{#2}}
% \newcommand{\gameVarPlayer}[3]{x_{#3},{#1}_{#3}^{#2}, \mathbf{u}_{#3}^{-#2}}
% \newcommand{\myinner}[1]{\langle(#1)x,x\rangle}


\newcommand{\usequence}[2]{(u_{i,t})_{i=1,t=1}^{#1,#2}}
\newcommand{\contTilde}[1]{\mathbf{\tilde{#1}}}
\newcommand{\transpose}{\mathsf{T}}
\newcommand{\myinner}[1]{\langle(#1)x,x\rangle}
\newcommand{\quadinner}[1]{x^{\transpose}(#1)x}
\DeclareMathOperator{\contB}{\mathbf{B}}
\DeclareMathOperator{\Log}{\mathrm{Log}}
\newcommand{\BK}[1]{\mathbf{B}\bar{K}_{#1}}
%bound \| x_t - x_{t|t} \|
\newcommand{\boundxtxtgt}[2]{C_{fb}^{2}\|\bar{x}_{1}\|[\frac{C_{K}}{\gamma-1}\bigg(\frac{1-(\eta\gamma)^{#1}}{1-\eta\gamma} - \frac{1-\eta^{#1}}{1-\eta} \bigg){#2}+\varepsilon_{K}\bigg(\frac{(#1-1)\eta^{#1+1}-#1\eta^{#1}+\eta}{(1-\eta)^{2}}\bigg)]}
\DeclareMathOperator{\tempRBP}{R + B^{\transpose}PB}

% \newcommand{\boundxtxtgt1}[1]{\frac{#1}{2}}
%bound |K_t|t - K_t^*|
\newcommand{\boundKt}{C_{K^{'}}\gamma^{W-1} + \gamma_{K^{'}}}

%bound xt|t
\newcommand{\boundxtgt}[1]{C_{fb}\eta^{#1}}

% \newcommand{\\det}{\mathit{\det}}
% \newcommand{\Log}[1]{\text{Log}}

\newtheorem{problem}{Problem}
\newtheorem{lemma}{Lemma}
\newtheorem{assumption}{Assumption}
\newtheorem{definition}{Definition}
\newtheorem{corollary}{Corollary}
\newtheorem{proposition}{Proposition}
\newtheorem{remark}{Remark}
\newtheorem{theorem}{Theorem}


\title{On the Regret Analysis of Online Feedback Potential Dynamic Game via Trajectory Prediction and Tracking}
\author{Yitian Chen}
% \date{June 2023}

\begin{document}

\maketitle

\section{INTRODUCTION}
Consider the linear systems
\begin{align}
    &x_{t+1} = Ax_{t} + \mathbf{B}\mathbf{u}_{t}\label{eq:linsys},\\
    &x_{1} =\bar{x}_{1}(\bar{x}_1 \in \mathbb{R}^{n})\label{eq:initialx}
\end{align}
where $t$ is an nonegative integer, $m$ and $n$ are positive integers, $A \in \mathbb{R}^{n\times n}$, $\mathbf{B} := [B^{1}, B^{2}]$, $B^{i} \in \mathbb{R}^{n\times m}(1\leq i \leq 2)$, $\mathbf{u}_{t} = [u_{1,t}^{\transpose},u_{2,t}^{\transpose}]^{\transpose}$, $x_{t}\in\mathbb{R}^n$ and $u_{i,t} \in \mathbb{R}^{m}$($1\leq t \leq T,1\leq i \leq 2$). 
Given $\mathbf{H}$ with 2 rows partitions and 2 column partitions, i.e.,
% Suppose matrix $\mathbf{H}$ is expressed as a collection of block matrices, i.e.,
\begin{align*}
    \mathbf{H} = 
    \begin{bmatrix}
        H_{11} & H_{12}\\
        H_{21} & H_{22}
    \end{bmatrix}.
\end{align*}
For each $i,j(1\leq i,j\leq 2)$, we define the notation
\begin{equation}
    [\mathbf{H}]_{ij} := H_{ij}.
\end{equation}


Moreover, define the following notations
\begin{align}
    &(u_{i,t})_{i=1,t=1}^{2,T-1} := (u_{1,1},u_{2,1},\cdots, u_{1,T-1},u_{2,T-1}),\\
    \begin{split}
         &(u_{i,t})_{i=1,t=1}^{2,\tau-1} \frown (v_{i,t})_{i=1,t=\tau}^{2,T-1}:=(u_{1,1},\cdots,u_{1,\tau-1},v_{1,\tau},\cdots,\\
    &\qquad v_{1,T-1},u_{2,1},\cdots,u_{2,\tau-1},v_{2,\tau},\cdots,v_{2,T-1} ),
    \end{split}
   \\ 
   & R_{t}^{i} := 
   \begin{bmatrix}
       [R_{t}^{i}]_{11} & [R_{t}^{i}]_{12}\\
       [R_{t}^{i}]_{21} & [R_{t}^{i}]_{22}
   \end{bmatrix}(1\leq i \leq 2),\\
    &g_{i,t}(x_{t}, \mathbf{u}_{t}) := \frac{1}{2}(x_{t}^{\mathsf{T}}Q_{t}^{i}x_{t} + 
    \mathbf{u}_{t}^{\transpose}R_{t}^{i}\mathbf{u}_{t})(1\leq i \leq 2),\\
    &g_{i,T}(x) := \frac{1}{2} x^{\mathsf{T}}Q_{T}^{i}x(1\leq i \leq 2).
\end{align}
For all $1 \leq t \leq T$ and $1\leq i\leq 2$, the matrices $Q_{t}^{i}$, $[R_{t}^{i}]_{pq}(1\leq p,q \leq 2)$, $A$ and $B^{i}(1\leq i \leq 2)$ satisfy the following assumptions
\begin{assumption}\label{assumption:bounds}
    There exists symmetric positive definite matrices $Q_{min}, Q_{max}, R_{min}, R_{max}$ that
    \begin{equation}
        Q_{min} \preceq Q_{t} \preceq Q_{max},
    \end{equation}
    \begin{equation}
        Q_{t} \in \mathbb{S}^{n}_{++},
    \end{equation}
    for $1\leq t \leq T$, and
    \begin{equation}\label{eq:positiveR}
        R_{min} \preceq 
        \begin{bmatrix}
            [R_{t}^{1}]_{11} & [R_{t}^{1}]_{12}\\
            [R_{t}^{2}]_{21} & [R_{t}^{2}]_{22}
        \end{bmatrix}
        \preceq R_{max},
    \end{equation}
    % \begin{equation}
    %     [R_{t}^{p}]_{pq} \in \mathbb{S}^{n}_{++},
    % \end{equation}
    % \begin{equation}
    %     [R_{t}^{p}]_{p,q} \preceq \frac{1}{4}R_{max},
    % \end{equation}
    % \begin{equation}
    %     [R_{t}^{1}]_{12}^{\transpose} = [R_{t}^{2}]_{21},
    % \end{equation}
    % \begin{equation}\label{eq:positiveR}
    %     R_{min} \preceq [R_{t}^{2}]_{22} - [R_{t}^{2}]_{21}([R_{t}^{1}]_{11})^{-1}[R_{t}^{1}]_{12},
    % \end{equation}
    for $1 \leq t \leq T-1$.
\end{assumption}
\begin{assumption}\label{assumption:controllable}
    Matrix $A$ has full rank, and there exists $\bar{K}$, such that $\rho(A + \contB \bar{K}) < 1$, where $\rho(\cdot)$ denotes the spectral radius operator.
\end{assumption}
Define the cost function for the $i$-th player,
\begin{equation}\label{eq:LQcost}
    J_{i,T}((x_{t})_{t=1}^{T},(u_{i,t})_{i=1,t=1}^{N,T-1}) := \sum_{t=1}^{T-1} g_{i,t}(x_{t}, \mathbf{u}_{t}) + g_{i,T}(x_{T}).
\end{equation}

Before we start discussing feedback Nash equilibrium, we use $(u_{i,t}^{*})_{i=1,t=1}^{2,T-1}$ to denote the control sequence for feedback Nash equilibrium, and $(u_{i,t})_{i=1,t=1}^{2,T-1}$ is a sequence that consisted by any given control vectors for each player. 

Furthermore, for $1 \leq \tau \leq T$, $(x_{t}^{*\tau})_{t=1}^{T}$ is generated by the sequence of $(u_{i,t})_{i=1,t=1}^{2,\tau-1} \frown (u_{i,t}^{*})_{i=1,t=\tau}^{2,T}$, where $u_{i,t}$ is any given control decision for $1 \leq i \leq 2$ and $1 \leq t \leq \tau-1$. Moreover, for $1 \leq \tau \leq T$, $(x_{t}^{(1,\tau)})_{t=1}^{T}$ is generated by $(u_{i,t})_{i=1,t=1}^{2,\tau-1} \frown (u_{1,\tau},u_{2,\tau}^{*}) \frown (u_{i,t}^{*})_{i=1,t=\tau+2}^{2,T}$. Similarly to $(x_{t}^{(2,\tau)})_{t=1}^{T}$.

The sequence $( u_{i,t}^{*})_{i=1,t=1}^{2,T}$ satisfies that, for any fix $i$ that $1 \leq i \leq 2$, such 
sequence satisfies that 
\begin{equation}\label{eq:nashIneq}
    \begin{split}
        \text{Level T}
        &\begin{cases}
            &J_{1,T}((x_{t}^{*T})_{t=1}^{T}, (u_{i,t})_{i=1,t=1}^{2,T-2} \frown (u_{1,T}^{*},u_{2,T}^{*})) \\ & \leq J_{1,T}((x_{t}^{(1,T)})_{t=1}^{T}, (u_{i,t})_{i=1,t=1}^{2,T-2} \frown (u_{1,T},u_{2,T}^{*})),\\ \\
            &J_{2,T}((x_{t}^{*T})_{t=1}^{T}, (u_{i,t})_{i=1,t=1}^{2,T-2} \frown (u_{1,T}^{*},u_{2,T}^{*})) \\ & \leq J_{2,T}((x_{t}^{(2,T)})_{t=1}^{T}, (u_{i,t})_{i=1,t=1}^{2,T-2} \frown (u_{1,T}^{*},u_{2,T})).
        \end{cases}
    \\ &\qquad \qquad \qquad \vdots \\
    \text{Level 1}
        &\begin{cases}
            &J_{1,T}((x_{t}^{*1})_{t=1}^{T}, (u_{1,1}^{*},u_{2,1}^{*}) \frown (u_{i,t}^{*})_{i=1,t=2}^{2,T-1}) \\ & \leq J_{1,T}((x_{t}^{(1,1)})_{t=1}^{T}, (u_{1,1},u_{2,1}^{*}) \frown (u_{i,t}^{*})_{i=1,t=2}^{2,T-1}),\\ \\
            &J_{2,T}((x_{t}^{*1})_{t=1}^{T}, (u_{1,1}^{*},u_{2,1}^{*}) \frown (u_{i,t}^{*})_{i=1,t=2}^{2,T-1}) \\ & \leq J_{2,T}((x_{t}^{(2,1)})_{t=1}^{T}, (u_{1,1}^{*},u_{2,1}) \frown (u_{i,t}^{*})_{i=1,t=2}^{2,T-1}).
        \end{cases}
    \end{split}
\end{equation}
For convenient use later, we define the notion of $\mathcal{H}_{t}$ and $\text{DFLGame}$ as
\begin{equation}\label{eq:history}
    \mathcal{H}_{T} := ( \bar{x}_{1},(Q_{t}^{i})_{i=1,t=1}^{2,T},(R_{t}^{i})_{i=1,t=1}^{2,T-1}),
\end{equation}
\begin{equation}\label{eq:regret}
 \text{DFLGame}(\mathcal{H}_{T},T):=((x_{t}^{*1})_{t=1}^{T}, (\mathbf{u}_{t}^{*})_{t=1}^{T-1}).
\end{equation}
% and the problem of finding such $(x_{t}^{*1})_{t=1}^{T}, (u_{i,t}^{*})_{i=1,t=1}^{2,T-1}$ is called a linear quadratic dynamic feedback game(LQDFG).
For any decisions $(\mathbf{u}_{t})_{t=1}^{T-1}$ and the associated state sequence $(x_{t})_{t=1}^{T}$, the dynamic social regret is defined as
\begin{equation}
    \begin{split}
        &\text{Regret}_{T}((\mathbf{u}_{t})_{t=1}^{T-1}) := \\
        &\sum_{i=1}^{N} J_{i,T}((x_{t})_{t=1}^{T},(\mathbf{u}_{t})_{t=1}^{T-1}) - J_{i,T}((x_{t}^{*})_{t=1}^{T},(\mathbf{u}_{t}^{*})_{t=1}^{T-1}).
    \end{split}
\end{equation}
The main focus of our work is to propose a novel control policy that generates $\mathbf{u}_{t}$ using the information available at time $t$, i.e., $\mathcal{H}_{t}$, and investigate its performance. We specifically consider a feedback control policy $\pi(\cdot,\cdot)$ of the form 
\begin{equation}\label{eq:policyForm}
    \mathbf{u}_{t} = \pi(x_{t}, \mathcal{H}_{t}).
\end{equation}

The key contributions of this paper are
\begin{itemize}
    \item The proposal of a method to solve an online LQ potential game;
    \item Development of the dynamic social regret upperbound;
\end{itemize}


%%%%%%%%%%%%%%%%%%%%%%%%%%%%%%%%%%%%%%%%%%%%%%%%%%%%%%%%%%%%%%%%%%%%%%%%%%%%%%%%
\section{PROBLEM FORMULATION}


In this paper, we consider the following problem.
\begin{problem}[Online Dynamic LQ Game]
     Consider the controllable system \eqref{eq:linsys}. Let the cost matrices in \eqref{eq:regret} satisfy Assumption \ref{assumption:controllable} for any given $T \geq 1$ and $W < T-1$. At time $1 \leq t \leq T-W-1$, the available information to the players is given by $\mathcal{H}_{t}$ as defined in \eqref{eq:history} and the current state $x_{t}$. It is desired to design a control policies $\pi(\cdot, \cdot)$ of the form \eqref{eq:policyForm} that yields a regret, as defined by \eqref{eq:regret}.
     % , that is independent of the bounds given in Assumption \ref{assumption:bounds}.
\end{problem}

\section{APPROACH AND REGRET ANALYSIS}
\paragraph{Prediction. } 
Define 
\begin{equation}
    \bar{\mathcal{H}}_{t} = (\bar{x}_{1}, (Q_{\tau})_{\tau=1}^{t+W} \frown(Q_{t+W})_{\tau=t+W+1}^{T}, (R_{\tau}^{i})_{i=1,\tau=1}^{2,t+W} \frown(R_{t+W}^{i})_{i=1,\tau=t+W+1}^{2,T-1}).
\end{equation}
We predict a trajectory by using the current cost matrices and setting the future unknown cost matrices to be equal to the known value at time $t+W$, i.e.,
\begin{equation}
    ((x_{\tau|t})_{\tau=1}^{T},(\mathbf{u}_{\tau|t})_{\tau=1}^{T-1}) = \text{DFLGame}(\bar{\mathcal{H}}_{t},T)
\end{equation}

\paragraph{Prediction Tracking. } We propose the following feedback control policy
\begin{equation}\label{eq:policy}
    \pi(x_{t},\mathcal{H}_{t}) := \bar{K}(x_{t}-x_{t|t}) + \mathbf{u}_{t|t},
\end{equation}
where $\bar{K}\in \mathbb{R}^{m\times n}$ is a control matrix such that $\rho(A+\mathbf{B}\bar{K}) < 1$, and $\rho(\cdot)$ denotes the matrix spectral radius.


\subsection{Regret Analysis}
\begin{definition}[LQDFG]
    A linear quadratic dynamic feedback game(LQDFG) is, for each player $i(1\leq i\leq 2)$ and a given positive integer $T$, suppose $J_{i,t}(\cdot,\cdot)$ for $1\leq i\leq 2,1\leq t\leq T$ are known by each player, where $J_{i,t}$ is defined at \eqref{eq:LQcost}. At stage $t(1\leq t\leq T-1)$, each player chooses the control $u_{i,t}^{*}$ that satisfies the inequalities given in \eqref{eq:nashIneq}. 
\end{definition}

% Before we introduce the concept of linear quadratic dynamic feedback potential game(LQDFPG), let's start by introducing the following linear quadratic optimal control problem(LQOCP). For a given $T(T\geq 1)$, 
% suppose $\Psi_{t}:\mathbb{R}^{n}\times\mathbb{R}^{m}\rightarrow \mathbb{R}(1\leq t \leq T-1)$ and $\Psi_{T}:\mathbb{R}^{n}\rightarrow \mathbb{R}$ are being continuous and twice continuously differentiable in their arguments.
Under Assumption \ref{assumption:controllable}, LQOCP is defined by the following problem.

\begin{definition}[LQOCP]
    The problem of linear quadratic optimal control problem(LQOCP) is the following.
    Suppose $T$ is a positive integer. For given $(\bar{Q}_{t})_{t=1}^{T},(\bar{R}_{t})_{t=1}^{T-1}$ satisfy (...existence of solution), find $\{\mathbf{u}_{t}\}_{t=1}^{T-1}$ such that
    \begin{equation}\label{eq:LQOCP}
\begin{split}
    \min_{\{\mathbf{u}_{t}\}_{t=1}^{T-1}} \sum_{t=1}^{T-1} x_{t}^{\transpose}\bar{Q}_{t}x_{t} + \mathbf{u}_{t}^{\transpose}\bar{R}_{t}\mathbf{u}_{t} + x_{T}^{\transpose}\bar{Q}_{T}x_{T} \text{\qquad subject to Equation \eqref{eq:linsys}}.
\end{split}
\end{equation}
\end{definition}



\begin{definition}
    The dynamic game is referred to as a linear quadratic dynamic feedback potential game(LQDFPG), if the solution of a given LQOCP, in feedback form, provides a feedback Nash equilibrium for the linear quadratic dynamic game. 
\end{definition}

\begin{lemma}
    For time instant $t$ from $T-1$ to 1, define
        \begin{equation}\label{eq:Theta}
        [\Theta_{t}]_{ij} := [R_{t}]^{i}_{ij} + B^{i\transpose}P_{t+1}^{i}B^{j},
        \end{equation}
    where $R_{t}^{1},R_{t}^{2},Q_{t}$ from a LQDFG satisfy
        \begin{equation}\label{eq:costFPDG1}
            [R_{t}^{1}]_{12} + B^{1\transpose}P_{t+1}^{1}B^{2} = ([R_{t}^{2}]_{21} + B^{2\transpose}P_{t+1}^{2}B^{1})^{\transpose}(1\leq t\leq T-1),
        \end{equation}
        \begin{equation}\label{eq:costFPDG2}
            \Theta_{t} \succ 0(1\leq t \leq T-1),
        \end{equation}
        \begin{equation}\label{eq:costFPDG3}
            \mathbf{B}^{\transpose}(P_{t}^{1}-P_{t}^{2})A=0(1\leq t \leq T),
        \end{equation}
    Then, the LQDFG that has such parameters is a linear quadratic dynamic feedback potential game(LQDFPG).
\end{lemma}
The above lemma is a special case of \cite[Theorem 6]{prasad_structure_2023} when $Q_{t}^{1}=Q_{t}^{2}$ for $t(1\leq t \leq T)$.

\begin{lemma}\label{lemma:gamePGrelation}
    \begin{equation}
        [R_{t}^{i}]_{ii} + B^{i\transpose}P_{t+1}^{i}B^{i} = [\bar{R}_{t}]_{ii} + B^{i\transpose}\bar{P}_{t+1}B^{i},
    \end{equation}
    \begin{equation}
        B^{i\transpose}P_{t+1}^{i}A = B^{i\transpose}\bar{P}_{t+1}A,
    \end{equation}
    \begin{equation}
        [R_{t}^{i}]_{ij} + B^{i\transpose}P_{t+1}^{i}B^{j} = [\bar{R}_{t}]_{ij} + B^{i\transpose}\bar{P}_{t+1}B^{j},i\neq j,
    \end{equation}
    \begin{equation}
        [R_{t}^{1}]_{12} + B^{1\transpose}P_{t+1}^{1}B^{2} = [R^{2}_{t}]_{21} + B^{2\transpose}P^{2}_{t+1}B^{1},
    \end{equation}
    for $i,j=1,2$. Then, ...
\end{lemma}


\begin{lemma}
    \cite[Theorem 5]{prasad_structure_2023}
    For a given positive integer $T$, define
    \begin{align*}
        \Omega := ((\bar{Q}_{1},\bar{Q}_{2},\cdots,\bar{Q}_{T},\bar{R}_{1},\cdots,\bar{R}_{T-1})|\mathcal{K}),
    \end{align*}
    where $\mathcal{K}$ is the conditions that, at time instant $T$, $\bar{Q}_{T}$ is such that
        \begin{equation}
            \mathbf{B}^{\transpose}\bar{Q}_{T}A = \mathbf{B}^{\transpose}Q_{T}A,
        \end{equation}
        and, let $\bar{P}_{T}$ satisfy
        \begin{equation}
            \mathbf{B}^{\transpose}\bar{P}_{T}A = \mathbf{B}^{\transpose}Q_{T}A.
        \end{equation}
        Moreover, set $\bar{R}_{t}$ as
        \begin{equation}\label{eq:matrixR}
            \bar{R}_{t} = \Theta_{t} - \mathbf{B}^{\transpose}\bar{P}_{t+1}\mathbf{B},
        \end{equation}
        where $\Theta_{t} \succ 0$,
        with $\bar{Q}_{t}$ such that
        \begin{equation}
            \bar{Q}_{t} = Q_{t} + K_{t}^{\transpose}(R_{t}^{1}-\bar{R}_{t})K_{t},
        \end{equation}
        where
        \begin{equation}
            K_{t} = \Theta_{t}^{-1}
            \begin{bmatrix}
                B_{t}^{1\transpose}P_{t+1}^{1}\\
                B_{t}^{2\transpose}P_{t+1}^{2}
            \end{bmatrix}
            A,
        \end{equation}
        from $t=T-1$ to $t=1$.
    Then, every element in $\Omega$ is an associated LQOCP for the LQDFPG. Further, $\Omega$ is non-empty.
\end{lemma}

\begin{corollary}
    % There exists matrices $\bar{R}_{min},\bar{R}_{max} \in \mathbf{S}_{++}^{m}$, that the cost matrix defined in \eqref{eq:matrixR} satisfy the following
    Matrix $\bar{R}_{t}$ defined in \eqref{eq:matrixR} satisfy
    \begin{equation}
        0 \prec R_{min} \preceq \bar{R}_{t} \preceq R_{max},
    \end{equation}
    for $1\leq t \leq T-1$.
\end{corollary}
\begin{proof}
    By definition of $\bar{R}_{t}$, we have
    \begin{align*}
        \bar{R}_{t} &= 
        \begin{bmatrix}
            [R_{t}^{1}]_{11} & [R_{t}^{1}]_{12}\\
            [R_{t}^{2}]_{21} & [R_{t}^{2}]_{22}
        \end{bmatrix}
        + 
        \begin{bmatrix}
            B^{1\transpose}P_{t+1}^{1}\\
            B^{2\transpose}P_{t+1}^{2}
        \end{bmatrix}\mathbf{B}
        - \mathbf{B}^{\transpose}\bar{P}_{t+1}\mathbf{B}\\
        &= \begin{bmatrix}
            [R_{t}^{1}]_{11} & [R_{t}^{1}]_{12}\\
            [R_{t}^{2}]_{21} & [R_{t}^{2}]_{22}
        \end{bmatrix}
        + 
        (\begin{bmatrix}
            B^{1\transpose}P_{t+1}^{1}\\
            B^{2\transpose}P_{t+1}^{2}
        \end{bmatrix}-
        \begin{bmatrix}
            B^{1\transpose}\bar{P}_{t+1}\\
            B^{2\transpose}\bar{P}_{t+1}
        \end{bmatrix}
        )\mathbf{B}
    \end{align*}
    Due to Assumption \ref{assumption:controllable}, $A$ is full-rank. Therefore, for $1\leq i \leq 2$,
    \begin{align*}
        B^{i\transpose}P_{t+1}^{i} = B^{i\transpose}\bar{P}_{t+1}.
    \end{align*}
    Thus,
    \begin{align*}
        \bar{R}_{t} &= 
        \begin{bmatrix}
            [R_{t}^{1}]_{11} & [R_{t}^{1}]_{12}\\
            [R_{t}^{2}]_{21} & [R_{t}^{2}]_{22}
        \end{bmatrix}.
    \end{align*}
    By using Equation \eqref{eq:positiveR} from Assumption \ref{assumption:bounds}, we can arrived with the conclusion.
    % Suppose $x\in \mathbb{R}^{2m}$ is non-zero. Let $x = [x_{1:m} | x_{m+1:2m}]$, we have
    % \begin{align*}
    %     &x^{\transpose}\begin{bmatrix}
    %         [R_{t}^{1}]_{11} & [R_{t}^{1}]_{12}\\
    %         [R_{t}^{2}]_{21} & [R_{t}^{2}]_{22}
    %     \end{bmatrix}x \\
    %     &= x_{1:m}^{\transpose}[R_{t}^{1}]_{11}x_{1:m} + x_{m+1:2m}^{\transpose}[R_{t}^{2}]_{22}x_{m+1:2m} + x_{1:m}^{\transpose}[R_{t}^{1}]_{12}x_{m+1:2m} + x_{m+1:2m}^{\transpose}[R_{t}^{2}]_{21}x_{1:m}
    % \end{align*}
    % By using the properties of eigenvalues, we have
    % \begin{align*}
    %     \lambda_{min}([R_{t}^{1}]_{11}) + \lambda_{min}([R_{t}^{2}]_{22}) \leq x_{1:m}^{\transpose}[R_{t}^{1}]_{11}x_{1:m} + x_{m+1:2m}^{\transpose}[R_{t}^{2}]_{22}x_{m+1:2m} \leq \lambda_{max}([R_{t}^{1}]_{11}) + \lambda_{max}([R_{t}^{2}]_{22}).
    % \end{align*}
    % Moreover, define $y = U_{[R_{t}^{1}]_{12}}x_{1:m}$ and $z = V_{[R_{t}^{1}]_{12}}^{*}x_{m+1:2m}$,
    % \begin{align*}
    %     x_{1:m}^{\transpose}[R_{t}^{1}]_{12}x_{m+1:2m} &= \sum_{i=1}^{m} \lambda_{i}([R_{t}^{1}]_{12}x_{m+1:2m}) y_{i}z_{i} \\
    %     &\geq 0\\
    %     &\leq \lambda_{max}([R_{t}^{1}]_{12}).
    % \end{align*}
    % Similar procedures can be applied to matrix $[R_{t}^{2}]_{21}$. Therefore, 
    % \begin{align*}
    %     x^{\transpose}\begin{bmatrix}
    %         [R_{t}^{1}]_{11} & [R_{t}^{1}]_{12}\\
    %         [R_{t}^{2}]_{21} & [R_{t}^{2}]_{22}
    %     \end{bmatrix}x
    % \end{align*}
    % has finite eigenvalues. We can thus construct such $\bar{R}_{min}$ and $\bar{R}_{max}$ that
    % \begin{align*}
    %     0 \prec \bar{R}_{min} \preceq \bar{R}_{t} \preceq \bar{R}_{max},
    % \end{align*}
    % by eigen-decomposition of $\bar{R}_{t}$.
\end{proof}

\begin{remark}
    Based on the proof above, for any given $\tau,t(1\leq \tau \leq t \leq T-1)$, we have
    \begin{equation*}
        \bar{R}_{\tau|t} = \bar{R}_{\tau}.
    \end{equation*}
\end{remark}

\begin{lemma}\label{lemma:matrixK}
    Suppose $m \leq n$ and $R\in \mathbb{S}^{m}_{++}$. 
    If $P\in \mathbb{S}^{n}_{++}$, then for any given $B$ that has appropriate size and finite singular values, all singular values of matrix $K = (R+B^{\transpose}PB)^{-1}B^{\transpose}P$ is lower and upper bounded by the following inequality,
    \begin{equation}
        0 \leq \sigma_{k}(K) < \theta_{k},
    \end{equation}
    where $\sigma_{k}$ denotes the $k$-th singular value of $K$, $\sigma_{1}(K) \geq \cdots \geq \sigma_{m}(K)$, and $\theta_{k} \in (\sigma_{\min}(B),\sigma_{\max}(B))$.
\end{lemma}
\begin{proof}
    Define an $n\times n$ matrix
    \begin{equation}
        \underbar{B} := 
        \begin{bmatrix}
            B & 0_{n\times n-m}
        \end{bmatrix},
    \end{equation}
    and suppose
    \begin{equation}
        \lambda_{1}(B^{\transpose}K^{\transpose}KB) \geq \cdots \geq \lambda_{m}(B^{\transpose}K^{\transpose}KB).
    \end{equation}
    Note that
    \begin{align*}
       0 =  \lambda_{m+1}(B^{\transpose}K^{\transpose}KB) = \cdots = \lambda_{n}(B^{\transpose}K^{\transpose}KB),
    \end{align*}
    due to $\textit{rank}(B^{\transpose}K^{\transpose}KB) \leq m$.
    By using \cite[Theorem 4.5.9]{horn_matrix_2013} (see also the discussion on \cite[p.~284]{horn_matrix_2013}), for any $k(1\leq k\leq m\leq n,\text{where $\textit{dim}(B)=m$})$,
    we have that
    \begin{align*}
        \lambda_{k}(B^{\transpose}K^{\transpose}KB) &= \lambda_{k}(\underbar{B}^{\transpose}K^{\transpose}K\underbar{B})\\
        &= \theta_{k}\lambda_{k}(K^{\transpose}K),
    \end{align*}
    where $0\leq \theta_{k} \leq \sigma_{max}(\underbar{B})=\sigma_{max}(B)$.
    Due to
    \begin{align*}
        \sigma_{k}(KB) &= \sigma_{k}((R+B^{\transpose}PB)^{-1}B^{\transpose}PB)\\
        &= \sigma_{k}\bigg((I - (R+B^{\transpose}PB)^{-1}R) \bigg),
    \end{align*}
    and
    \begin{align*}
       0 \leq \sigma_{k}\bigg((I - (R+B^{\transpose}PB)^{-1}R) \bigg) < 1,
    \end{align*}
    when $P \in \mathbb{S}_{++}^{n}$. This is due to the following. Consider a scalar $\sigma$ and matrix
    \begin{align*}
        &G :=\\
        &(I-(\tempRBP)^{-1}R)^{\transpose}(I-(\tempRBP)^{-1}R) - (1-\sigma) I \\
        &= \sigma I - R(\tempRBP)^{-1} - (\tempRBP)^{-1}R + R(\tempRBP)^{-1}(\tempRBP)^{-1}R\\
        &= R\bigg(\sigma R^{-2} - (\tempRBP)^{-1}R^{-1} - R^{-1}(\tempRBP)^{-1} + (\tempRBP)^{-1}(\tempRBP)^{-1} \bigg)R\\
        &= R(\tempRBP)^{-1}\bigg(\sigma(\tempRBP)R^{-1}R^{-1}(\tempRBP) - R^{-1}(\tempRBP) - (\tempRBP)R^{-1} + I \bigg)(\tempRBP)^{-1}R\\
        &= R(\tempRBP)^{-1}\bigg(\sigma(I+B^{\transpose}PBR^{-1})(I+R^{-1}B^{\transpose}PB) - R^{-1}(\tempRBP) - (\tempRBP)R^{-1} + I \bigg)(\tempRBP)^{-1}R\\
        &= R(\tempRBP)^{-1}\bigg((\sigma-1)(I+B^{\transpose}PBR^{-1}+R^{-1}B^{\transpose}PB) + \sigma(B^{\transpose}PBR^{-1}R^{-1}B^{\transpose}PB)\bigg)(\tempRBP)^{-1}R.
    \end{align*}
    Therefore, if $\det(G) = 0$, then there exist zero eigenvalue of $(\sigma-1)(I+B^{\transpose}PBR^{-1}+R^{-1}B^{\transpose}PB) + \sigma(B^{\transpose}PBR^{-1}R^{-1}B^{\transpose}PB)$.

    Since $B^{\transpose}PBR^{-1}R^{-1}B^{\transpose}PB = (R^{-1}B^{\transpose}PB)^{\transpose}(R^{-1}B^{\transpose}PB)(:= G^{'})$, therefore
    \begin{align*}
        \lambda_{k}(B^{\transpose}PBR^{-1}R^{-1}B^{\transpose}PB) \geq 0(1\leq k \leq m).
    \end{align*}
    Moreover, Let $\bar{G} := I+B^{\transpose}PBR^{-1}+R^{-1}B^{\transpose}PB$, 
    Suppose $\lambda_{1}(\bar{G}) \geq \cdots \geq \lambda_{m}(\bar{G})$. By Weyl's inequality, for $k(1\leq k \leq m)$, we have
    \begin{equation}\label{eq:eigenStep3}
        \lambda_{k}(I) + \lambda_{m}(B^{\transpose}PBR^{-1}+R^{-1}B^{\transpose}PB) \leq \lambda_{k}(\bar{G}),
    \end{equation}
    
    by applying Weyl's inequality again, we have
    \begin{equation}\label{eq:eigenStep4}
        0 < 1 \leq \lambda_{k}(I) + \lambda_{m}(B^{\transpose}PBR^{-1})+\lambda_{m}(R^{-1}B^{\transpose}PB) \leq \lambda_{k}(\bar{G}).
    \end{equation}

     We can further conclude that
     \begin{align}
         (\sigma-1)\lambda_{k}(G^{'})+\sigma\lambda_{m}(\bar{G}) \leq \lambda_{k}(G) \leq (\sigma-1)\lambda_{k}(G^{'})+\sigma\lambda_{1}(\bar{G}).
     \end{align}
     If $\sigma \geq 1$, due to
     \begin{align}
         (\sigma-1)\lambda_{k}(G^{'})+\sigma\lambda_{m}(\bar{G}) \leq \lambda_{k}(G) \leq (\sigma-1)\lambda_{k}(G^{'})+\sigma\lambda_{1}(\bar{G}),
     \end{align}
     we have
     \begin{align*}
         (\sigma-1)\lambda_{k}(G^{'})+\sigma\lambda_{m}(\bar{G}) > 0,
     \end{align*}
     therefore $\lambda_{k}(G) > 0(\forall k)$. If $\sigma < 0$, 
     \begin{align*}
         &(\sigma-1)\lambda_{k}(G^{'})+\sigma\lambda_{m}(\bar{G}) < 0,\\
         &\lambda_{k}(G) \leq (\sigma-1)\lambda_{k}(G^{'})+\sigma\lambda_{1}(\bar{G}) < 0,
     \end{align*}
     therefore $\lambda_{k}(G) < 0(\forall k)$. Thus, $0 \leq \sigma < 1$.

    Thus, all singular values of matrix $I - (R+B^{\transpose}PB)^{-1}R$ are in between 0 and 1, we can conclude that
    \begin{align}
        0 \leq \sigma_{k}(K) < \theta_{k},
    \end{align}
    for $1\leq k \leq n$.
    
    % \begin{align*}
    %     \det(R(R+B^{\transpose}PB)^{-1}(R+B^{\transpose}PB)^{-1}R) > 0,
    % \end{align*}
    % and
    % \begin{equation}\label{eq:eigenStep1}
    % \begin{split}
    %     &\det(R(R+B^{\transpose}PB)^{-1}(R+B^{\transpose}PB)^{-1}R - I)\\
    %     &= \det(R)\det([(R+B^{\transpose}PB)^{-1}]^{2} - [R^{-1}]^{2})\det(R)\\
    %     &= \det(R)^{2}\det(-R^{-1}(R+B^{\transpose}PB)^{-1}-(R+B^{\transpose}PB)^{-1}R^{-1}+[(R+B^{\transpose}PB)^{-1}]^{2})\\
    %     &= \det(R)^{2}\det((R+B^{\transpose}PB)^{-1})^{2}\det(-(R+B^{\transpose}PB)R^{-1}-R^{-1}(R+B^{\transpose}PB)+I)\\
    %     &= \det(R)^{2}\det((R+B^{\transpose}PB)^{-1})^{2}\det(-(I + R^{-1}B^{\transpose}PB + B^{\transpose}PBR^{-1})).
    % \end{split}
    % \end{equation}
    % Suppose $\lambda\in (-\infty,0]$, the determinant
    % \begin{equation}\label{eq:eigenStep2}
    %     \begin{split}
    %         \det(R^{-1}B^{\transpose}PB - \lambda I) = \det(R^{-1})\det(B^{\transpose}PB - \lambda R) \geq 0,
    %     \end{split}
    % \end{equation}
    % therefore, the solution of $\lambda$ is in the region of $[0,\infty)$. The steps from Equation \eqref{eq:eigenStep1} to Equation \eqref{eq:eigenStep4} can be apply to matrix $B^{\transpose}PBR^{-1}$. Let $G := I + R^{-1}B^{\transpose}PB + B^{\transpose}PBR^{-1}$. Suppose $\lambda_{1}(G) \geq \cdots \geq \lambda_{m}(G)$. By Weyl's inequality, for $k(1\leq k \leq m)$, we have
    % \begin{equation}\label{eq:eigenStep3}
    %     \lambda_{k}(I) + \lambda_{m}(B^{\transpose}PBR^{-1}+R^{-1}B^{\transpose}PB) \leq \lambda_{k}(G),
    % \end{equation}
    
    % by applying Weyl's inequality again, we have
    % \begin{equation}\label{eq:eigenStep4}
    %     1 \leq \lambda_{k}(I) + \lambda_{m}(B^{\transpose}PBR^{-1})+\lambda_{m}(R^{-1}B^{\transpose}PB) \leq \lambda_{k}(G).
    % \end{equation}
    % Thus, $\lambda_{k}(R(R+B^{\transpose}PB)^{-1}(R+B^{\transpose}PB)^{-1}R - I) < 0$ for $1\leq k \leq m$. 
    % The above suggests that, there exists $\sigma^{'} < 0, \sigma > 0$, such that
    % \begin{align*}
    %     \det(G - (1+\sigma^{'})I) &= 0,\\
    %     \det(G - \sigma I) &= 0.
    % \end{align*}
    % Let $\sigma^{'} = \sigma - 1$, we have that $\sigma-1 < 0$, thus $\sigma < 1$. Therefore,
    % \begin{align*}
    %     0 < \sigma < 1.
    % \end{align*}
    
    % % $\det(G) > 0$, and $\det(R(R+B^{\transpose}PB)^{-1}(R+B^{\transpose}PB)^{-1}R - I) < 0$. By mean value theorem, there exists a scalar $\sigma(0 < \sigma < 1)$ that satisfies
    % % \begin{align*}
    % %     \det(R(R+B^{\transpose}PB)^{-1}(R+B^{\transpose}PB)^{-1}R - \sigma I) = 0.
    % % \end{align*}
    % % Moreover, due to matrix $R(R+B^{\transpose}PB)^{-1}(R+B^{\transpose}PB)^{-1}R - \sigma I$ is positive definite for $\sigma \in (-\infty,0]$. Its determinants are negative for $\sigma \in (1,\infty)$ if we repeat the steps from Equation \eqref{eq:eigenStep1} to Equation \eqref{eq:eigenStep4}. 
    
    % Thus, all singular values of matrix $I - (R+B^{\transpose}PB)^{-1}R$ are in between 0 and 1, we can conclude that
    % \begin{align}
    %     0 \leq \sigma_{k}(K) < \theta_{k},
    % \end{align}
    % for $1\leq k \leq n$.
\end{proof}

\begin{corollary}\label{corrolary:boundedK}
    Suppose the cost matrices $(Q_{t})_{t=1}^{T}$ and $(R_{t}^{i})_{t=1,i=1}^{T-1,2}$ satisfy the conditions described from Equation \eqref{eq:costFPDG1} to \eqref{eq:costFPDG3} and Assumption \ref{assumption:bounds}.
    For a given preview horizon $W(0\leq W \leq T-1)$, define
    \begin{equation}
        Q_{\tau|t}:= 
        \begin{cases}
            Q_{\tau} \text{ if $1\leq \tau \leq t+W$}\\
            Q_{t+W} \text{ if $t+W < \tau \leq T$},
        \end{cases}
    \end{equation}
    and
    \begin{equation}
        R_{\tau|t}^{i}:= 
        \begin{cases}
            R_{\tau}^{i} \text{ if $1\leq \tau \leq t+W$}\\
            R_{t+W}^{i} \text{ if $t+W < \tau \leq T$},
        \end{cases}
    \end{equation}
    for $i = {1,2}$.
    There exists a scalar $\omega_{K}$ such that
    \begin{equation*}
        \|K_{\tau|t}\| \leq \omega_{K},
    \end{equation*}
    for $1\leq \tau,t\leq T$.
\end{corollary}

\begin{proof}
    This can be directly concluded by letting $\omega_{K} = \sigma_{max}(B)\sigma_{max}(A)$ and applying Lemma \ref{lemma:matrixK}.
\end{proof}

\begin{assumption}\label{assumption:lowerQ}
    For any given $t(1\leq t \leq T-1)$, matrices $Q_{t}, R_{t}^{1}, \bar{R}_{t}$ satisfy
    \begin{equation}
        \lambda_{\min}(Q_{t}) > \max(0,\sigma_{max}(B)\lambda_{min}(\bar{R}_{t}-R_{t}^{1}))
    \end{equation}
\end{assumption}

\begin{corollary}
    For any given $\tau,t(1\leq \tau, t \leq T-1)$, matrix
    \begin{equation}
        \bar{Q}_{\tau|t} = Q_{\tau|t} + K_{\tau|t}^{\transpose}(R_{\tau|t}^{1} - \bar{R}_{\tau|t})K_{\tau|t}
    \end{equation}
    is positive definite.
\end{corollary}
\begin{proof}
    Suppose
    \begin{align*}
        &\lambda_{1}(Q_{\tau|t}) \geq \cdots \geq \lambda_{n}(Q_{\tau|t}),\\
        &\lambda_{1}(K_{\tau|t}^{\transpose}(R_{\tau|t}^{1} - \bar{R}_{\tau|t})K_{\tau|t}) \geq \cdots \geq \lambda_{n}(K_{\tau|t}^{\transpose}(R_{\tau|t}^{1} - \bar{R}_{\tau|t})K_{\tau|t}),\\
        &\lambda_{1}(R_{\tau|t}^{1} - \bar{R}_{\tau|t}) \geq \cdots \geq \lambda_{n}(R_{\tau|t}^{1} - \bar{R}_{\tau|t}).
    \end{align*}
    By Weyl's inequality,
    \begin{align*}
        \lambda_{min}(\bar{Q}_{\tau|t}) &= \lambda_{n}(\bar{Q}_{\tau|t})\\
        &\geq \lambda_{n}(Q_{\tau|t}) + \lambda_{n}(K_{\tau|t}^{\transpose}(R_{\tau|t}^{1} - \bar{R}_{\tau|t})K_{\tau|t}).
    \end{align*}
    By \cite{horn_matrix_2013}[Theorem 4.5.9], there exists a positive scalar $\theta_{n}$ such that
    \begin{equation}
        \lambda_{n}(K_{\tau|t}^{\transpose}(R_{\tau|t}^{1} - \bar{R}_{\tau|t})K_{\tau|t}) = \theta_{n}\lambda_{n}(R_{\tau|t}^{1} - \bar{R}_{\tau|t}) = \theta_{n}\lambda_{min}(R_{\tau|t}^{1} - \bar{R}_{\tau|t}),
    \end{equation}
    where $\sigma_{min}(K_{\tau|t}) \leq \theta_{n} \leq \sigma_{max}(K_{\tau|t}) < \sigma_{max}(B)$.
    The last inequality is due to Lemma \ref{lemma:matrixK}. Moreover,
    \begin{align*}
        \theta_{n}\lambda_{min}(R_{\tau|t}^{1} - \bar{R}_{\tau|t}) \geq \max(0,\sigma_{B}\lambda_{min}(R_{\tau|t}^{1} - \bar{R}_{\tau|t})).
    \end{align*}
    Thus, 
    \begin{align*}
        \lambda_{min}(\bar{Q}_{\tau|t}) &= \lambda_{min}(Q_{\tau|t}+K_{t}^{\transpose}(R_{\tau|t}^{1}-\bar{R}_{\tau|t})K_{\tau|t}) \\
        &> \lambda_{min}(Q_{\tau|t}) +\max(0,\sigma_{B}\lambda_{min}(R_{\tau|t}^{1} - \bar{R}_{\tau|t})) > 0.
    \end{align*}
    Therefore, $\lambda_{min}(\bar{Q}_{\tau|t})$ is positive definite.
\end{proof}

\begin{remark}
    Assumption \ref{assumption:lowerQ} suggests that, if the distance between the cost matrices $R_{t}^{2}$ and $R_{t}^{1}$ is sufficiently small, or $R_{t}^{1}$ is a dominant cost than $R_{t}^{2}$, then matrix $Q_{\tau|t}$ are positive definite. This will help us to develop the contraction between $K_{\tau|t}$ and $K_{\tau|t_{0}}$ for ($1\leq \tau \leq t,t_{0} \leq T-1$).
\end{remark}
\begin{assumption}
    An LQDFG with parameters $\{Q_{\tau|t}\}_{\tau=1}^{T}$ and $\{R_{\tau|t}\}_{\tau=1}^{T-1}$ is an LQDFPG.
\end{assumption}
\begin{lemma}\label{lemma:boundedP}
    Define
    \begin{equation}
        \bar{P}_{T} = \bar{Q}_{T},
    \end{equation}
    for $t = T$, and 
    \begin{equation}
    \begin{split}
        &\bar{K}_{t} = -(R_{t}+ B^{\transpose}\bar{P}_{t+1}B)^{-1}B^{\transpose}\bar{P}_{t+1},\\
        &\bar{P}_{t} = \bar{Q}_{t} + \bar{K}_{t}^{\transpose}\bar{R}_{t}\bar{K}_{t} + (A+B\bar{K}_{t})^{\transpose}\bar{P}_{t+1}(A+B\bar{K}_{t}),
    \end{split}
    \end{equation}
    for $1\leq t \leq T-1$, recursively. There exists positive definite matrices $\bar{P}_{min}$ and $\bar{P}_{max}$ such that
    \begin{equation}
        0 \prec \bar{P}_{min} \preceq \bar{P}_{t} \preceq \bar{P}_{max}.
    \end{equation}
\end{lemma}
\begin{proof}
    By Corrollary \ref{corrolary:boundedK} and Assumption \ref{assumption:lowerQ}, there exists positive definite matrices $\bar{Q}_{min},\bar{Q}_{max}$ such that
    \begin{equation}
        \bar{Q}_{min} \preceq \bar{Q}_{t} \preceq \bar{Q}_{max},
    \end{equation}
    for $1\leq t \leq T$.
    Together with the Assumption \ref{assumption:bounds}, by following a similar procedure as \cite[Proposition 11.]{zhang_regret_2021}, there exists positive definite $\bar{P}_{min}$ and $\bar{P}_{max}$ such that
    \begin{equation*}
        \bar{P}_{min} \preceq \bar{P}_{t} \preceq \bar{P}_{max}.
    \end{equation*}
\end{proof}


\begin{lemma}\label{lemma:m}
    Suppose matrices $T,V_{1},V_{2},S$ are positive definite. If
    \begin{equation}
        m \geq 1+\frac{\lambda_{max}(V_{1}-V_{2})}{\lambda_{min}(T+V_{2})},
    \end{equation}
    then for any non-zero $x$, we have
    \begin{equation}
        \frac{\quadinner{T+V_{1}}}{\quadinner{S+V_{2}}} \leq m\frac{\quadinner{T+V_{2}}}{\quadinner{S+V_{2}}}
    \end{equation}
\end{lemma}
\begin{proof}
    By the assumption of $m$, for any nonzero $x$, we have
    \begin{align*}
        m &\geq 1+\frac{\lambda_{max}(V_{1}-V_{2})}{\lambda_{min}(T+V_{2})}\\
        &\geq 1+ \frac{\quadinner{V_{1}-V_{2}}}{\quadinner{T+V_{2}}}\\
        &= \frac{\quadinner{T+V_{2}}}{\quadinner{T+V_{2}}} + \frac{\quadinner{V_{1}-V_{2}}}{\quadinner{T+V_{2}}}\\
        &= \frac{\quadinner{T+V_{1}}}{\quadinner{T+V_{2}}}.
     \end{align*}
     Due to $\quadinner{S+V_{2}} > 0$ and $T,V_{2},V_{1}$ are positive definite, we have that
     \begin{align*}
         m\frac{\quadinner{T+V_{2}}}{\quadinner{S+V_{2}}} \geq \frac{\quadinner{T+V_{1}}}{\quadinner{S+V_{2}}}.
     \end{align*}
\end{proof}



% \begin{lemma}[Not sure again]
%     If the above lemma is true, we expect there exists $\bar{Q}_{min},\bar{Q}_{max}(\|\bar{Q}_{min}\|,\|\bar{Q}_{max}\|<\infty)$ that
%     \begin{equation*}
%         \bar{Q}_{min} \preceq \bar{Q}_{t} \preceq \bar{Q}_{max},
%     \end{equation*}
%     for $1\leq t \leq T$. Which
%     \begin{equation*}
%         \bar{Q}_{t} = Q_{t}+ \mathbf{K}_{t}^{\transpose}(R_{t}^{1}-\bar{R}_{t})\mathbf{K}_{t},
%     \end{equation*}
%     $\Theta_{t}$ is defined by \eqref{eq:Theta}
%     and $\mathbf{K}_{t}$ is defined by
%     \begin{equation*}
%         \mathbf{K}_{t} = -(\Theta_{t}^{-1})
%         \begin{bmatrix}
%             B^{1\transpose}P_{t+1}^{1} \\
%             B^{2\transpose}P_{t+1}^{2}
%         \end{bmatrix}A.
%     \end{equation*}
%     Which
%     \begin{equation*}
%         P_{t}^{i} = Q_{t} + \mathbf{K}_{t}^{\transpose}R_{t}^{i}\mathbf{K}_{t} + (A+\mathbf{B}\mathbf{K}_{t})^{\transpose}P_{t+1}^{i}(A+\mathbf{B}\mathbf{K}_{t}),
%     \end{equation*}
%     for $i = (1,2)$, $1\leq t \leq T-1$, and
%     \begin{equation*}
%         P_{T}^{i} = Q_{T},
%     \end{equation*}
%     for $i = (1,2)$,
%     recursively.
% \end{lemma}

\begin{definition}
    For any $X,Y\in \mathbb{S}_{++}^{n}$, define the operator $\delta_{\infty}(\cdot, \cdot)$ as
    \begin{equation*}
        \delta_{\infty}(X, Y) := \| \log(Y^{-\frac{1}{2}}XY^{-\frac{1}{2}})\|.
        % \delta_{\infty}(X, Y) := (\sum_{i=1}^{d} \log^{2}(\lambda_{i}))^{\frac{1}{2}},
    \end{equation*}
    % where $\lambda_{1},\cdots, \lambda_{d}$ are the eigenvalues of the matrix $XY^{-1}$.
\end{definition}

\begin{proposition}\label{proposition:deltaRatio}
    For any $X,Y\in \mathbb{S}_{++}^{n}$,
    \begin{equation}
        \delta_{\infty}(X,Y) = \log(\sup_{\xi\neq0} \frac{\xi^{\transpose}X\xi}{\xi^{\transpose}Y\xi}).
    \end{equation}
\end{proposition}

\begin{remark}\label{remark:delta}
    For $T,S,V_{1},V_{2}\in \mathbb{S}_{++}^{n}$. Based on Lemma \ref{lemma:boundedP},\ref{lemma:m} and Proposition \ref{proposition:deltaRatio}, suppose a positive scalar $m$ satisfy
    \begin{align*}
        m \geq 1+\frac{\lambda_{max}(V_{1}-V_{2})}{\lambda_{min}(T+V_{2})}.
    \end{align*}
    We have
    \begin{align*}
        \delta_{\infty}(T+V_{1},S+V_{2}) &= \log( \sup_{x\neq 0} \frac{\quadinner{T+V_{1}}}{\quadinner{S+V_{2}}})\\
        &\leq \log(\sup_{x\neq 0} \frac{\quadinner{T+V_{1}}}{\quadinner{S+V_{2}}})\\
        &\leq \log(m)+\log(\sup_{x\neq 0} \frac{\quadinner{T+V_{2}}}{\quadinner{S+V_{2}}})\\
        &\leq \log(m) + \delta_{\infty}(T+V_{2},S+V_{2})\\
        &\leq \log(m) + \gamma\delta_{\infty}(T,S),
    \end{align*}
    where $\gamma(0<\gamma<1)$ can be found by \cite{krauth_finite-time_2019}[Lemma D.2].
\end{remark}

\begin{lemma}\label{lemma:boundedPK}
    For any $\tau,t,t_{0}(1\leq \tau\leq t\leq t_{0}\leq T)$, suppose the preview window length is given by an non-negative integer $W(W\leq T-1)$. There exists scalars $\gamma,\varepsilon_{P},\varepsilon_{K},C_{P},C_{K}^{'}(0<\gamma<1,\varepsilon>0,C_{P}>0,C_{K}^{'}>0)$, such that
    \begin{equation}
        \|\bar{P}_{\tau|t}-\bar{P}_{\tau|t_{0}}\| \leq C_{P}\gamma^{t-\tau+W}+\varepsilon_{P},
    \end{equation}
    and
    \begin{equation}
        \|\bar{K}_{\tau|t}-\bar{K}_{\tau|t_{0}}\| \leq C_{K}^{'}\gamma^{t-\tau+1+W}+\varepsilon_{K}^{'}.
    \end{equation}
\end{lemma}
\begin{proof}
     For any $\tau,t,t_{0}(1\leq \tau\leq t\leq t_{0}\leq T)$, by \cite[Lemma D.2]{krauth_finite-time_2019}, Lemma \ref{lemma:boundedP}, \ref{lemma:m} and Remark \ref{remark:delta}. 
     Let
     \begin{align*}
        &\alpha_{\tau,t} := \lambda_{max}(A^{\transpose}(\bar{P}_{\tau+1|t}^{-1}+B\bar{R}_{\tau|t}^{-1}B^{\transpose})^{-1}A)\\
        &\beta_{\tau,t} := \lambda_{min}(\bar{Q}_{\tau|t})\\ 
        &\gamma := \max_{1\leq \tau,t \leq T} \frac{\alpha_{\tau,t}}{\alpha_{\tau,t}+\beta_{\tau,t}}, 
     \end{align*}
     we have
    \begin{align*}
        \delta_{\infty}(\bar{P}_{\tau|t},\bar{P}_{\tau|t_{0}}) &= \delta_{\infty}(\bar{Q}_{\tau|t}+A^{\transpose}(\bar{P}_{\tau+1|t}^{-1}+B\bar{R}_{\tau|t}^{-1}B^{\transpose})^{-1}A,\\
        & \qquad \bar{Q}_{\tau|t_{0}}+A^{\transpose}(\bar{P}_{\tau+1|t_{0}}^{-1}+B\bar{R}_{\tau|t_{0}}^{-1}B^{\transpose})^{-1}A)\\
        % &\leq \delta_{\infty}(Q_{\tau}+K_{\tau|t}^{\transpose}(R_{\tau}^{1}-\bar{R}_{\tau})K_{\tau|t}+A^{\transpose}(P_{\tau+1|t}^{-1}+B\bar{R}_{\tau|t}^{-1}B^{\transpose})^{-1}A,\\
        % & \qquad Q_{\tau}+K_{\tau|t_{0}}^{\transpose}(R_{\tau}^{1}-\bar{R}_{\tau})K_{\tau|t_{0}}+A^{\transpose}(P_{\tau+1|t_{0}}^{-1}+B\bar{R}_{\tau|t_{0}}^{-1}B^{\transpose})^{-1}A)\\
        &\leq \gamma\delta_{\infty}(\bar{P}_{\tau+1|t},\bar{P}_{\tau+1|t_{0}}) + \varepsilon_{1}\\
        &\leq \gamma^{t-\tau+W}\delta_{\infty}(\bar{P}_{t+W|t},\bar{P}_{t+W|t_{0}}) + \varepsilon_{1}(1+\cdots+\gamma^{t-\tau-1+W})\\
        &< \gamma^{t-\tau+W}\delta_{\infty}(\bar{P}_{t+W|t},\bar{P}_{t+W|t_{0}}) + \frac{\varepsilon_{1}}{1-\gamma}\\
        &\leq C_{P1}\gamma^{t-\tau+W}+\frac{\varepsilon_{1}}{1-\gamma},
    \end{align*}
    where $\varepsilon_{1}$ can be found as similar as the term $\log(m)$ from Remark \ref{remark:delta}, and $C_{P1} = \max_{t\leq t_{0}} \delta_{\infty}(\bar{P}_{t|t},\bar{P}_{t|t_{0}})$ . By inequality
    \begin{align*}
        \frac{e^{x}-1}{x} \leq \frac{e^{c}-1}{c},
    \end{align*}
    for $0 < x \leq c$. Let $h = \max_{(\tau,t_{0}| \tau,\leq t_{0})} \delta_{\infty}(\bar{P}_{\tau|t},\bar{P}_{\tau|t_{0}})$, we have
    \begin{equation}
        \frac{exp(\delta_{\infty}(\bar{P}_{\tau|t},\bar{P}_{\tau|t_{0}}))-1}{\delta_{\infty}(\bar{P}_{\tau|t},\bar{P}_{\tau|t_{0}})} \leq \frac{exp(h)-1}{h},
    \end{equation}
    then
    \begin{align*}
        \|\bar{P}_{\tau|t}-\bar{P}_{\tau|t_{0}}\| &\leq \lambda_{max}(\bar{P}_{\tau|t_{0}})\frac{\exp(h)-1}{h}\delta_{\infty}(\bar{P}_{\tau|t},\bar{P}_{\tau|t_{0}})\\
        &< C_{P}\gamma^{W}+\varepsilon_{P},
    \end{align*}
    where $C_{P} = \frac{C_{P1}\lambda_{max}(\bar{P}_{\tau|t_{0}})(\exp(h)-1)}{h}$ and $\varepsilon_{P} = \frac{\varepsilon_{1}\lambda_{max}(\bar{P}_{\tau|t_{0}})(\exp(h)-1)}{h(1-\gamma)}$. Following by similar steps of equations (20) and (21) from \cite[Lemma 8]{chen_regret_2022}, we can find $C_{K}^{'},\varepsilon_{K}^{'}(C_{K}^{'}>0,\varepsilon_{K}^{'}>0)$ we have that
    \begin{align*}
        \|\bar{K}_{\tau|t}-\bar{K}_{\tau|t_{0}}\| \leq C_{K}^{'}\gamma^{t-\tau-1+W}+\varepsilon_{K}^{'}.
    \end{align*}
\end{proof}

\begin{remark}
    When $\tau = t$ and $t_{0} = T$, we have
    \begin{align*}
        \|\bar{K}_{t|t}-\bar{K}_{t}^{*}\| \leq C_{K}^{'}\gamma^{W-1} + \varepsilon_{K}^{'}
    \end{align*}
\end{remark}

\begin{lemma}\label{lemma:multGain}
    For time horizon $T$, suppose $1\leq t_{0} \leq t_{1}\leq t\leq T-1$. There exists scalars $C_{fb},\eta(C_{fb}>0,0<\eta<1)$ such that
    \begin{equation}
        \left \| \prod_{\tau=t_{0}}^{t_{1}}(A+\mathbf{B}\bar{K}_{\tau|t})  \right\| \leq C_{fb}\eta^{t_{1}-t_{0}+1}.
    \end{equation}
\end{lemma}
This lemma can be proved following the same steps as those found in the proof of \cite[Appendix E,Proposition 2]{zhang_regret_2021}.
\begin{lemma}
    Suppose the preview window horizon is given by a non-negative integer $W(W \leq T-1)$
    \begin{equation}
        \|x_{t}-x_{t|t}\| \leq C_{fb}^{2}\|\bar{x}_{1}\|q^{t}[\frac{C_{K}}{\gamma-1}\bigg(\frac{1-(\frac{\eta\gamma}{q})^{t}}{1-\frac{\eta\gamma}{q}} - \frac{1-(\frac{\eta}{q})^{t}}{1-\frac{\eta}{q}} \bigg)+\varepsilon_{K}\bigg(\frac{(t-1)(\frac{\eta}{q})^{t+1}-t(\frac{\eta}{q})^{t}+\frac{\eta}{q}}{(1-\frac{\eta}{q})^{2}}\bigg)]
    \end{equation}
\end{lemma}
\begin{proof}
    Suppose $N$ is a positive integer. For an arbitrary sequence $(a_{i})_{i=1}^{N}$. Suppose $e$ is the identity element for the sequence. For $1\leq p,q\leq N$, define the product operator as
\begin{equation}
    \prod_{j=p}^{q} a_{j} := 
    \begin{cases}
        a_{q}a_{q-1}\dots a_{p} & \text{if $p < q$}\\
        a_{q} & \text{if $p = q$}\\
        e & \text{if $p > q$},
    \end{cases}
\end{equation}
Define $\omega_{t} := x_{t}-x_{t|t}$, $\theta_{\tau| p,q} := x_{\tau|p}-x_{\tau|q}$, where $\tau \leq p\leq q\leq T$. Consequently, $w_{1} = 0$ and $\theta_{1|p,q}=0$. We now investigate the dynamics of $w_{t}$ and $\theta_{\tau|p,q}$. For integer $\tau > 1$, we note that
\begin{align*}
    &\theta_{\tau|p,q}\\
    &= x_{\tau|p}-x_{\tau|q}\\
    &= (A+\sum_{j}^{2}B^{j}K_{j,\tau|p})x_{\tau-1|p} - (A+\sum_{j}^{2}B^{j}K_{j,\tau|q})x_{\tau-1|q}\\
    &= (A+\sum_{j}^{2}B^{j}K_{j,\tau|p})\theta_{\tau-1|p,q} + [\sum_{j=1}^{2} B^{j}(K_{j,\tau-1|p}-K_{j,\tau-1|q})]x_{\tau-1|q}\\
    &\qquad \vdots\\
    &= \sum_{i=1}^{\tau-1}\bigg(\prod_{j=i+1}^{\tau-1}(A+\sum_{m=1}^{2}B^{m}K_{m,j|p})\bigg)\bigg[\sum_{m=1}^{2}B^{m}(K_{m,i|p}-K_{m,i|q})\bigg]x_{i|q}  \\
    &= \sum_{i=1}^{\tau-1}\bigg(\prod_{j=i+1}^{\tau-1}(A+\sum_{m=1}^{2}B^{m}K_{m,j|p})\bigg)\bigg[\sum_{m=1}^{2}B^{m}(K_{m,i|p}-K_{m,i|q})\bigg]\bigg(\prod_{n=1}^{i-1} (A+\sum_{m=1}^{2}B^{m}K_{m|q})\bigg)\bar{x}_{1}\\
    &= \sum_{i=1}^{\tau-1}\bigg(\prod_{j=i+1}^{\tau-1}(A+\mathbf{B}\bar{K}_{j|p})\bigg)\bigg[\mathbf{B}^{\mathsf{T}}(\bar{K}_{j|p}^{\mathsf{T}}-\bar{K}_{j|q}^{\mathsf{T}})\bigg]\bigg(\prod_{n=1}^{i-1} (A+\mathbf{B}\bar{K}_{n|q})\bigg)\bar{x}_{1}.
\end{align*}
Thus, for integer $t(1 < t\leq T)$, we have
\begin{align*}
    \omega_{t} = x_{t} - x_{t|t} &= Ax_{t-1} + \sum_{j=1}^{2}B^{j}[K^{j}(x_{t-1}-x_{t-1|t-1})+K_{t-1|t-1}^{j}x_{t-1|t-1}] - x_{t|t}\\
    &= (A+\sum_{j=1}^{2}B^{j}K^{j})x_{t-1} + [\sum_{j=1}^{2}B^{j}(K_{t-1|t-1}^{j}-K^{j})]x_{t-1|t-1} - x_{t|t}\\
    &= (A+\sum_{j=1}^{2}B^{j}K^{j})(x_{t-1}-x_{t-1|t-1}) + x_{t|t-1}-x_{t|t}\\
    &= \sum_{i=1}^{t} (A+\sum_{j=1}^{2}B^{j}K^{j})^{t-i} \theta_{i|i-1,i}.
\end{align*}

We now investigate the dynamics of $\theta_{\tau|p,q}$. Note that $\theta_{0|p,q} = 0$, and 

\begin{align*}
    \theta_{\tau+1|p,q} &= x_{\tau+1|p} - x_{\tau+1|q}\\
    &= (A+\BK{\tau|p})x_{\tau|p}-(A+\BK{\tau|q})x_{\tau|q}\\
    &= (A+\BK{\tau|p})(\theta_{\tau|p,q}+x_{\tau|q})-(A+\BK{\tau|q})x_{\tau|q}\\
    &= (A+\BK{\tau|p})\theta_{\tau|p,q} + \mathbf{B}(\bar{K}_{\tau|p}-\bar{K}_{\tau|q})x_{\tau|q}.
\end{align*}
This implies that
\begin{align*}
    x_{\tau+1|p} - x_{\tau+1|q} = \sum_{n=1}^{\tau}\bigg(\prod_{m=n+1}^{\tau}(A+\BK{m|p})\bigg)\mathbf{B}(\bar{K}_{n|p}-\bar{K}_{n|q})\bigg(\prod_{m=1}^{n-1}(A+\BK{m|p})\bigg)\bar{x}_{1}.
\end{align*}
By Lemma \ref{lemma:multGain}, we can bound the product term by
\begin{equation*}
    \| \prod_{m=n+1}^{\tau}(A+\BK{m|p})\| \leq C_{fb}\eta^{\tau-n},
\end{equation*}
By Lemma \ref{lemma:boundedPK}, let $C_{K} := \|\mathbf{B}\|C_{K}^{'}$ and $\varepsilon_{K} := \|\mathbf{B}\|\varepsilon_{K}^{'}$, we have
\begin{equation*}
    \|\mathbf{B}(\bar{K}_{n|p}-\bar{K}_{n|q})\| \leq C_{K}\gamma^{p-n+W}+\varepsilon_{K}.
\end{equation*}
Thus,
\begin{align*}
    \|\theta_{\tau+1|p,q}\| &= \|x_{\tau+1|p}-x_{\tau+1|q}\|\\
    &\leq C_{fb}^{2}\|\bar{x}_{1}\|(\sum_{n=1}^{\tau}\eta^{\tau}(C_{K}\gamma^{p-n}+\varepsilon_{K}))\\
    &= C_{fb}^{2}[\frac{C_{K}\gamma^{p+W}\eta^{\tau}}{1-\frac{1}{\gamma}}(1-(\frac{1}{\gamma})^{\tau+1})+ \varepsilon_{K}\tau\eta^{\tau}].
\end{align*}
Choosing $\tau = t,p = t$ and $q = T$, this results in
\begin{align*}
    \|\theta_{t|t,T}\| &\leq C^{2}_{fb}\|\bar{x}_{1}\|[\frac{C_{K}\gamma^{t+W}\eta^{t}}{1-\frac{1}{\gamma}}(1-(\frac{1}{\gamma})^{t+1}) + \varepsilon_{K}t\eta^{t}]\\
    &= C^{2}_{fb}\|\bar{x}_{1}\|[\frac{C_{K}\gamma^{1+W}\eta^{t}}{\gamma-1}(\gamma^{t}-1) + \varepsilon_{K}t\eta^{t}].
\end{align*}

Moreover,

\begin{align*}
    \|\theta_{i|i-1,i}\| \leq C_{fb}^{2}\|\bar{x}_{1}\|[\frac{C_{K}\eta^{i-1}\gamma^{W}}{\gamma-1}(\gamma^{i}-1) + \varepsilon_{K}(i-1)\eta^{i-1}].
\end{align*}
By similar procedure from the proof of \cite[Lemma 10]{chen_regret_2022} below Equation (25), we can find $q,C_{q}(C_{q}>0,0<q<1)$ such that $\|(A+\mathbf{B}\bar{K})^{t-i}\| \leq C_{q}q^{t-i}$. Conclude the above, we have
\begin{align*}
    \|x_{t}-x_{t|t}\| &\leq \sum_{\tau=1}^{t}\|(A+\mathbf{B}\bar{K})^{t-i}\theta_{i|i-1,i}\|\\
    &\leq C_{fb}^{2}C_{q}\|\bar{x}_{1}\|\sum_{i=1}^{t}q^{t-i}[\frac{C_{K}\eta^{i-1}\gamma^{W}}{\gamma-1}(\gamma^{i}-1) + \varepsilon_{K}(i-1)\eta^{i-1}]\\
    &= C_{fb}^{2}\|\bar{x}_{1}\|q^{t}[\frac{C_{K}}{\gamma-1}\bigg(\frac{1-(\frac{\eta\gamma}{q})^{t}}{1-\frac{\eta\gamma}{q}} - \frac{1-(\frac{\eta}{q})^{t}}{1-\frac{\eta}{q}} \bigg)+\varepsilon_{K}\bigg(\frac{(t-1)(\frac{\eta}{q})^{t+1}-t(\frac{\eta}{q})^{t}+\frac{\eta}{q}}{(1-\frac{\eta}{q})^{2}}\bigg)]
\end{align*}
% Thus,
% \begin{align*}
%     \|x_{t}-x_{t}^{*}\| \leq 
% \end{align*}
\end{proof}

\begin{lemma}[Cost Difference Lemma]
For positive integer $T$, suppose for $t$ and $i$ satisfy $1 \leq t \leq T$ and $1 \leq i \leq 2$. Let 
$\{\pi_{i,t}\}_{i=1,t=1}^{2,T-1}$,$\{\tilde{\pi}_{i,t}\}_{i=1,t=1}^{2,T-1}$ are policies that maps $\mathbb{R}^{n}$ to $\mathbb{R}^{m}$, and each $g_{i,t}$ maps $\mathbb{R}^{n} \times \mathbb{R}^{m}$ to $\mathbb{R}$. In the later presentations, we state the convention $\bar{\pi}_{t}^{t_{0}} := \{\pi_{i,\tau}\}_{i=1,\tau=t}^{2,t_{0}}$ for the indexes $i,t$.

Define $x_{t+1}^{\bar{\pi}_{t}^{t+1}}(x_{t}):= Ax_{t} + \sum_{i=1}^{2} B^{i}\pi_{i,t}(x_{t})$,
\begin{align}
\begin{split}
    &V_{i,T,t}^{\bar{\pi}_{t}^{T-1}}(x_{t}) := \\
    &\begin{cases}
        \sum_{l=0}^{T-t} g_{i,t+l}(x_{t+1+l}^{\bar{\pi}_{t+l}^{t+l+1}}(x_{t+l}),\pi_{1,t+l}(x_{t+l}),\pi_{2,t+l}(x_{t+l})) & \text{$1 \leq t \leq T$}\\
        $0$ & \text{$t > T$},
    \end{cases}
\end{split}
    \\
    &Q_{i,T,t}^{\bar{\pi}_{t}^{T-1}}(x_{t},u_{1,t},u_{2,t}) := g_{t}(x_{t},u_{1,t},u_{2,t}) + V_{i,T,t+1}^{\bar{\pi}_{t+1}^{T-1}}(x_{t+1}^{\bar{\pi}_{t+1}^{T-1}})
    ,
\end{align}
Then we have the following equality,
\begin{align}
    J_{i,T}(\{x_{t}\}_{t=1}^{T},\{u_{1,t},u_{2,t}\}_{t=1}^{T-1}) - J_{i,T}(\{\tilde{x}_{t}\}_{t=1}^{T},\{\tilde{u}_{1,t},\tilde{u}_{2,t}\}_{t=1}^{T-1})\\
    = \sum_{t=1}^{T} Q_{i,T,t}^{\bar{\pi}_{t}^{T}}(x_{t},u_{1,t},u_{2,t}) -  V_{i,T,t}^{\bar{\pi}_{t}^{T}}(x_{t}),
\end{align}
where $\tilde{x}_{t+1} = \tilde{x}_{t+1}^{\bar{\pi}_{t}^{t+1}}(\tilde{x}_{t})$, $\tilde{x}_{1} = x_{1}$ and $x_{t+1},u_{1,t},u_{2,t}$ satisfy \eqref{eq:linsys}.
\end{lemma}

\begin{proof}
\begin{align*}
    \sum_{t=1}^{T-1} &Q_{i,T,t}^{\bar{\pi}_{t}^{T-1}}(x_{t},u_{1,t},u_{2,t}) -  V_{i,T,t}^{\bar{\pi}_{t}^{T-1}}(x_{t}) \\
    &= \sum_{t=1}^{T-1} g_{t}(x_{t},u_{1,t},u_{2,t})+ V_{i,T,t+1}^{\bar{\tilde{\pi}}_{t+1}^{T-1}}(x_{t+1})-V_{i,T,t}^{\bar{\tilde{\pi}}_{t}^{T-1}}(x_{t})\\
    &= \underbrace{g_{T}(x_{T}) + \sum_{t=1}^{T-1} g_{t}(x_{t},u_{1,t},u_{2,t})}_{J_{i,T}(\{x_{t}\}_{t=1}^{T},\{u_{1,t},u_{2,t}\}_{t=1}^{T-1})} - V_{i,T,1}^{\bar{\tilde{\pi}}_{1}^{T-1}}(x_{1})\\
    &= J_{i,T}(\{x_{t}\}_{t=1}^{T},\{u_{1,t},u_{2,t}\}_{t=1}^{T-1}) - J_{i,T}(\{\tilde{x}_{t}\}_{t=1}^{T},\{\tilde{u}_{1,t},\tilde{u}_{2,t}\}_{t=1}^{T-1})
\end{align*}
\end{proof}


\begin{corollary}\label{corollary:Delta}
    There exists positive scalars $\Delta_{1}$ and $\Delta_{2}$, such that
    \begin{align}
    \label{eq:delta1}
        &\|R_{t}^{i}+\mathbf{B}^{\transpose}P_{t+1}^{i}\mathbf{B}\| \leq \Delta_{1},\\
        \label{eq:delta2}
        &\|(R_{t}^{i}+\mathbf{B}^{\transpose}P_{t+1}^{i}\mathbf{B})\bar{K}_{t}^{*}+\mathbf{B}^{\transpose}P_{t+1}^{i}A\| \leq \Delta_{2}.
    \end{align}
\end{corollary}
\begin{proof}
    If costs $\{R_{t}^{i}\}_{i=1,t=1}^{2,T-1}$ and $\{Q_{t}\}_{t=1}^{2,T}$ from LQDFG is an LQDFPG. By Lemma \ref{lemma:gamePGrelation}, there exists a scalar $\Delta^{'}$ that is independent of $t$ and $i$, which
    \begin{align*}
        \|\contB^{\transpose}P_{t+1}^{i}\| = \|\contB^{\transpose}\bar{P}_{t+1}\| \leq \Delta^{'}.
    \end{align*}
    Due to Assumption \ref{assumption:bounds}, matrix $\|R_{t}^{i}\|$ is upperbounded uniformly w.r.t. $t$ and $i$. Thus we can find such $\Delta_{1}$ and $\Delta_{2}$ to satisfy equation \eqref{eq:delta1} - \eqref{eq:delta2}.
\end{proof}


\begin{theorem}
    \begin{align*}
        &\text{Regret}_{T}((\mathbf{u}_{t})_{t=1}^{T-1})\\
        &\leq \|\bar{x}_{1}\|^{2}\bigg[\Delta_{a}\frac{1-q^{2T}}{1-q^{2}} +  \Delta_{b}(\gamma^{W},\varepsilon_{K^{'}})\frac{1-\eta^{2T}}{1-\eta^{2}}+ \Delta_{c}(\gamma^{W},\varepsilon_{K^{'}})\frac{1-(q\eta)^{T}}{1-q\eta}\bigg].
    \end{align*}
\end{theorem}

\begin{proof}
Let $\{\pi_{i,t}\}_{i=1,t=1}^{2,T-1}$ be the control policies given in Equation \eqref{eq:policy}, and $\{\tilde{\pi}_{i,t}\}_{i=1,t=1}^{2,T-1}$ be the control policies that generate feedback Nash equilibrium.
By applying Cost Difference Lemma, we have
\begin{align*}
    &\text{Regret}_{T}((\mathbf{u}_{t})_{t=1}^{T-1})\\
    &= \frac{1}{2}\sum_{i=1}^{2}\sum_{t=1}^{T-1} Q_{i,T,t}^{\{\tilde{\pi}_{i,\tau}\}_{i=1,\tau=t}^{2,T-1}}(x_{t},u_{1,t},...,u_{N,t}) -  V_{i,T,t}^{\{\tilde{\pi}_{i,\tau}\}_{i=1,\tau=t}^{2,T-1}}(x_{t})\\
    &= \frac{1}{2}\sum_{i=1}^{2}\sum_{t=1}^{T-1} x_{t}^{\transpose}Q_{t}x_{t} + \mathbf{u}_{t}^{\transpose}R_{t}^{i}\mathbf{u}_{t} + (Ax_{t}+\mathbf{B}\mathbf{u}_{t})^{\transpose}P_{t+1}^{i}(Ax_{t}+\mathbf{B}\mathbf{u}_{t})\\
    &\qquad - x_{t}^{\transpose}Q_{t}x_{t} - \contTilde{u}_{t}^{\transpose}R_{t}^{i}\contTilde{u}_{t} - (Ax_{t}+\mathbf{B}\contTilde{u}_{t})^{\transpose}P_{t+1}^{i}(Ax_{t}+\mathbf{B}\contTilde{u}_{t})\\
    &= \frac{1}{2}\sum_{i=1}^{2}\sum_{t=1}^{T-1} \mathbf{u}_{t}^{\transpose}(R_{t}^{i}+\mathbf{B}^{\transpose}P_{t+1}^{i}\mathbf{B})\mathbf{u}_{t}-\contTilde{u}_{t}^{\transpose}(R_{t}^{i}+\mathbf{B}^{\transpose}P_{t+1}^{i}\mathbf{B})\contTilde{u}_{t} + 2x_{t}^{\transpose}A^{\transpose}P_{t+1}^{i}B(\mathbf{u}_{t}-\contTilde{u}_{t})\\
    &= \frac{1}{2}\sum_{i=1}^{2}\sum_{t=1}^{T-1} (\mathbf{u}_{t}-\contTilde{u}_{t})^{\transpose}(R_{t}^{i}+\mathbf{B}^{\transpose}P_{t+1}^{i}\mathbf{B})(\mathbf{u}_{t}-\contTilde{u}_{t})\\
    &\qquad + 2(\mathbf{u}_{t}-\contTilde{u}_{t})^{\transpose}\bigg((R_{t}^{i}+\mathbf{B}^{\transpose}P_{t+1}^{i}\mathbf{B})\contTilde{u}_{t}+\mathbf{B}^{\transpose}P_{t+1}^{i}Ax_{t}\bigg)\\
    &= \frac{1}{2}\sum_{i=1}^{2}\sum_{t=1}^{T-1} (\mathbf{u}_{t}-\contTilde{u}_{t})^{\transpose}(R_{t}^{i}+\mathbf{B}^{\transpose}P_{t+1}^{i}\mathbf{B})(\mathbf{u}_{t}-\contTilde{u}_{t})\\
    &\qquad + 2(\mathbf{u}_{t}-\contTilde{u}_{t})^{\transpose}\bigg((R_{t}^{i}+\mathbf{B}^{\transpose}P_{t+1}^{i}\mathbf{B})\bar{K}_{t}^{*}+\mathbf{B}^{\transpose}P_{t+1}^{i}A\bigg)x_{t}.
\end{align*}
Since
\begin{align*}
    \|\mathbf{u}_{t} - \contTilde{u}_{t}\|
    &= \|\bar{K}x_{t} + (\bar{K}_{t\mid t}-\bar{K})x_{t\mid t} - \bar{K}_{t}^{*}x_{t}\|\\
    &= \|(\bar{K}_{t}^{*}-\bar{K})(x_{t\mid t}-x_{t})  + (\bar{K}_{t|t}-\bar{K}_{t}^{*})x_{t|t}\|\\
    &\leq \|\bar{K}_{t}^{*}-\bar{K}\|\|x_{t\mid t}-x_{t}\|  + \|\bar{K}_{t|t}-\bar{K}_{t}^{*}\| \|x_{t|t}\|,
\end{align*}
and,
\begin{align*}
    \|\mathbf{u}_{t} - \contTilde{u}_{t}\|^{2}
    &= \|\bar{K}x_{t} + (\bar{K}_{t\mid t}-\bar{K})x_{t\mid t} - \bar{K}_{t}^{*}x_{t}\|^{2}\\
    &\leq 2(\|\bar{K}_{t}^{*}-\bar{K}\|^{2}\|x_{t\mid t}-x_{t}\|^{2}  + \|\bar{K}_{t|t}-\bar{K}_{t}^{*}\|^{2} \|x_{t|t}\|^{2}),
\end{align*}

By Corollary \ref{corollary:Delta}, there exists $\Delta_{1}>0,\Delta_{2}>0$, such that
\begin{align*}
    &\text{Regret}((\mathbf{u}_{t})_{t=1}^{T-1})\\
    &= \frac{1}{2}\sum_{i=1}^{2}\sum_{t=1}^{T-1} (\mathbf{u}_{t}-\contTilde{u}_{t})^{\transpose}(R_{t}^{i}+\mathbf{B}^{\transpose}P_{t+1}^{i}\mathbf{B})(\mathbf{u}_{t}-\contTilde{u}_{t})\\
    &\qquad + 2(\mathbf{u}_{t}-\contTilde{u}_{t})^{\transpose}\bigg((R_{t}^{i}+\mathbf{B}^{\transpose}P_{t+1}^{i}\mathbf{B})\bar{K}_{t}^{*}+\mathbf{B}^{\transpose}P_{t+1}^{i}A\bigg)x_{t}\\
    &\leq \sum_{t=1}^{T-1} (\mathbf{u}_{t}-\contTilde{u}_{t})^{\transpose}\bigg[\sum_{i=1}^{2}\frac{1}{2}(R_{t}^{i}+\mathbf{B}^{\transpose}P_{t+1}^{i}\mathbf{B})\bigg](\mathbf{u}_{t}-\contTilde{u}_{t})\\
    &\qquad + (\mathbf{u}_{t}-\contTilde{u}_{t})^{\transpose}\sum_{i=1}^{2}\bigg( (R_{t}^{i}+\mathbf{B}^{\transpose}P_{t+1}^{i}\mathbf{B})\bar{K}_{t}^{*}+\mathbf{B}^{\transpose}P_{t+1}^{i}A\bigg)x_{t}\\
    &\leq \Delta_{1} \sum_{t=1}^{T-1} \|\mathbf{u}_{t}-\contTilde{u}_{t}\|^2 + \Delta_{2}\sum_{t=1}^{T-1} \|\mathbf{u}_{t}-\contTilde{u}_{t}\|(\|x_{t}-x_{t|t}\| + \|x_{t|t}\|)\\
    &\leq 2\Delta_{1} \sum_{t=1}^{T-1} \bigg(\|\bar{K}_{t}^{*}-\bar{K}\|^{2}\|x_{t\mid t}-x_{t}\|^{2}  + \|\bar{K}_{t|t}-\bar{K}_{t}^{*}\|^{2} \|x_{t|t}\|^{2} \bigg)\\
    &\qquad + \Delta_{2}\sum_{t=1}^{T-1}\bigg( \|\bar{K}_{t}^{*}-\bar{K}\|\|x_{t\mid t}-x_{t}\|  + \|\bar{K}_{t|t}-\bar{K}_{t}^{*}\| \|x_{t|t}\|\bigg)\bigg(\|x_{t}-x_{t|t}\| + \|x_{t|t}\|\bigg).
\end{align*}
Due to
\begin{align*}
    &\|x_{t}-x_{t|t}\|\\
    &\leq C_{fb}^{2}\|\bar{x}_{1}\|q^{t}[\frac{C_{K}}{\gamma-1}\bigg(\frac{1-(\frac{\eta\gamma}{q})^{t}}{1-\frac{\eta\gamma}{q}} - \frac{1-(\frac{\eta}{q})^{t}}{1-\frac{\eta}{q}} \bigg)+\varepsilon_{K}\bigg(\frac{(t-1)(\frac{\eta}{q})^{t+1}-t(\frac{\eta}{q})^{t}+\frac{\eta}{q}}{(1-\frac{\eta}{q})^{2}}\bigg)]\\
    &\leq C_{x}\|\bar{x}_{1}\|q^{t}.
\end{align*}
To simplify the presentation of the main result, define
\begin{align*}
    &\Delta_{1} := \max_{t} \frac{1}{2}\sum_{i=1}^{2}(R_{t}^{i}+\mathbf{B}^{\transpose}P_{t+1}^{i}\mathbf{B}),\\
    &\Delta_{2} := \max_{t} \sum_{i=1}^{2}\bigg( (R_{t}^{i}+\mathbf{B}^{\transpose}P_{t+1}^{i}\mathbf{B})\bar{K}_{t}^{*}+\mathbf{B}^{\transpose}P_{t+1}^{i}A\bigg),\\
    &C_{x} := C_{fb}^{2}D_{K},\\
    &D_{K} := \frac{C_{K}q\eta(\gamma-1)}{(q-\eta\gamma)(q-\eta)} + \frac{\varepsilon_{K}q\eta}{(q-\eta)^{2}},\\
    &\Delta_{a} := 2(\Delta_{1}+\Delta_{2})D_{K}C_{x}^{2},\\
    &\Delta_{b}(z,y) := 4\Delta_{1}C_{fb}^{2}(C_{K}^{2}z^{2}+y^{2})+2\Delta_{2}(C_{K}z+y),\\
    &\Delta_{c}(z,y) := 2C_{x}C_{fb}\Delta_{2}(C_{K}z+y+D_{K}),
\end{align*}
we can upperbound the regret by
\begin{align*}
    &2\Delta_{1}\|\bar{x}_{1}\|^{2}\sum_{t=1}^{T-1}\bigg(D_{K}C_{x}^{2}q^{2t} + 2(C_{K}^{2}\gamma^{2W}+\varepsilon_{K^{'}}^{2})C_{fb}^{2}\eta^{2t}\bigg) \\
    &\leq 2\Delta_{1}\|\bar{x}_{1}\|^{2}\bigg(D_{K}C_{x}^{2}\frac{1-q^{2T}}{1-q^{2}} + 2C_{fb}^{2}(C_{K}^{2}\gamma^{2W}+\varepsilon_{K^{'}}^{2})\frac{1-\eta^{2T}}{1-\eta^{2}} \bigg),
\end{align*}
and
\begin{align*}
    &\sum_{t=1}^{T-1}\bigg( \|\bar{K}_{t}^{*}-\bar{K}\|\|x_{t\mid t}-x_{t}\|  + \|\bar{K}_{t|t}-\bar{K}_{t}^{*}\| \|x_{t|t}\|\bigg)\bigg(\|x_{t}-x_{t|t}\| + \|x_{t|t}\|\bigg)\\
    &\leq \|\bar{x}_{1}\|^{2}\sum_{t=1}^{T-1}\bigg(D_{K}C_{x}q^{t}+(C_{K^{'}}\gamma^{W}+\varepsilon_{K^{'}})C_{fb}\eta^{t}\bigg)(C_{x}q^{t}+C_{fb}\eta^{t})\\
    &\leq \|\bar{x}_{1}\|^{2}\bigg(D_{K}C_{x}^{2}\frac{1-q^{2T}}{1-q^{2}}+[D_{K}C_{x}C_{fb}+C_{x}C_{fb}(C_{K^{'}}\gamma^{W}+\varepsilon_{K^{'}})]\frac{1-(q\eta)^{T}}{1-q\eta}\\
    &+ C_{fb}^{2}(C_{K^{'}}\gamma^{W}+\varepsilon_{K^{'}})\frac{1-\eta^{2T}}{1-\eta^{2}} \bigg).
\end{align*}
Thus,
\begin{align*}
    &\text{Regret}_{T}((\mathbf{u}_{t})_{t=1}^{T-1})\\
    &\leq \|\bar{x}_{1}\|^{2}\bigg[ 2\Delta_{1}\bigg(D_{K}C_{x}^{2}\frac{1-q^{2T}}{1-q^{2}} + 2C_{fb}^{2}(C_{K}^{2}\gamma^{2W}+\varepsilon_{K^{'}}^{2})\frac{1-\eta^{2T}}{1-\eta^{2}} \bigg)\\
    &+2\Delta_{2}\bigg(D_{K}C_{x}^{2}\frac{1-q^{2T}}{1-q^{2}}+[D_{K}C_{x}C_{fb}+C_{x}C_{fb}(C_{K^{'}}\gamma^{W}+\varepsilon_{K^{'}})]\frac{1-(q\eta)^{T}}{1-q\eta}\\
    &+ C_{fb}^{2}(C_{K^{'}}\gamma^{W}+\varepsilon_{K^{'}})\frac{1-\eta^{2T}}{1-\eta^{2}} \bigg)\bigg]\\
    &= \|\bar{x}_{1}\|^{2}\bigg[\Delta_{a}\frac{1-q^{2T}}{1-q^{2}} +  \Delta_{b}(\gamma^{W},\varepsilon_{K^{'}})\frac{1-\eta^{2T}}{1-\eta^{2}}+ \Delta_{c}(\gamma^{W},\varepsilon_{K^{'}})\frac{1-(q\eta)^{T}}{1-q\eta}\bigg].
\end{align*}
\end{proof}

% \begin{figure}
%     \centering
%     \includegraphics[width=1\textwidth]{myfigure.pdf}
%     \caption{Regret for disturbance-free, 50 Monte-Carlo run, 2 players, time horizon 500, preview window horizon 501}
%     \label{fig:enter-label}
% \end{figure}

%%%%%%%%%%%%%%%%%%%%%%%%%%%%%%%%%%%%%%%%%%%%%%%%%%%%%%%%%%%%%%%%%%%%%%%%%%%%%%%%
\section{NUMERICAL SIMULATIONS}
\begin{figure}
        \label{fig:different_cases}
     \centering
     \subfigure[Regret vs. Time Horizon with 0 Preview Window Length]{
         \includegraphics[width=.45\textwidth]{window1.pdf}
    \label{fig:physSysReg}}
     \subfigure[Regret vs. Time Horizon with 1 Preview Window Length]{
         \includegraphics[width=.45\textwidth]{window2.pdf}
    \label{fig:ranSysReg}}
    \\
     \subfigure[Regret vs. Preview Window Length in 8-th Time Horizon]{
         \includegraphics[width=.45\textwidth]{time1.pdf}
    \label{fig:PhysSysDisReg}}
     \subfigure[Regret vs. Preview Window Length in 69-th Time Horizon]{
         \includegraphics[width=.45\textwidth]{timeThird.pdf}
    \label{fig:RanSysDisReg}}
    \caption{Performance measure $\text{Regret}_{T}$ for simulated systems.}
\end{figure}


\section{CONCLUSIONS AND FUTURE WORKS}

%%%%%%%%%%%%%%%%%%%%%%%%%%%%%%%%%%%%%%%%%%%%%%%%%%%%%%%%%%%%%%%%%%%%%%%%%%%%%%%%
\section{ACKNOWLEDGMENTS}

The authors gratefully acknowledge the contribution of National Research Organization and reviewers' comments.

\bibliographystyle{plain}
\bibliography{References}
\appendix
\end{document}


    
    
