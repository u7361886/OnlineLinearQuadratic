%%%%%%%%%%%%%%%%%%%%%%%%%%%%%%%%%%%%%%%%%%%%%%%%%%%%%%%%%%%%%%%%%%%%%%%%%%%%%%%%
%2345678901234567890123456789012345678901234567890123456789012345678901234567890
%        1         2         3         4         5         6         7         8

\documentclass[letterpaper, 10 pt, conference]{ieeeconf}  % Comment this line out
                                                          % if you need a4paper
%\documentclass[a4paper, 10pt, conference]{ieeeconf}      % Use this line for a4
                                                          % paper

\IEEEoverridecommandlockouts                              % This command is only
                                                          % needed if you want to
                                                          % use the \thanks command
\overrideIEEEmargins
% See the \addtolength command later in the file to balance the column lengths
% on the last page of the document



% The following packages can be found on http:\\www.ctan.org
\usepackage{graphics} % for pdf, bitmapped graphics files
\usepackage{epsfig} % for postscript graphics files
\usepackage{mathptmx} % assumes new font selection scheme installed
\usepackage{times} % assumes new font selection scheme installed
\usepackage{amsmath} % assumes amsmath package installed
\usepackage{amssymb}  % assumes amsmath package installed


\newcommand{\usequence}[2]{\{u_{i,t}\}_{i=1,t=1}^{#1,#2}}
\newcommand{\contTilde}[1]{\mathbf{\tilde{#1}}}
\newcommand{\transpose}{\mathsf{T}}

\newtheorem{problem}{Problem}
\newtheorem{lemma}{Lemma}
\newtheorem{assumption}{Assumption}
\newtheorem{definition}{Definition}
\newtheorem{corollary}{Corollary}

\title{\LARGE \bf
Preparation of Papers for IEEE CSS Sponsored Conferences \& Symposia
}

%\author{ \parbox{3 in}{\centering Huibert Kwakernaak*
%         \thanks{*Use the $\backslash$thanks command to put information here}\\
%         Faculty of Electrical Engineering, Mathematics and Computer Science\\
%         University of Twente\\
%         7500 AE Enschede, The Netherlands\\
%         {\tt\small h.kwakernaak@autsubmit.com}}
%         \hspace*{ 0.5 in}
%         \parbox{3 in}{ \centering Pradeep Misra**
%         \thanks{**The footnote marks may be inserted manually}\\
%        Department of Electrical Engineering \\
%         Wright State University\\
%         Dayton, OH 45435, USA\\
%         {\tt\small pmisra@cs.wright.edu}}
%}

\author{Huibert Kwakernaak and Pradeep Misra% <-this % stops a space
\thanks{This work was not supported by any organization}% <-this % stops a space
\thanks{H. Kwakernaak is with Faculty of Electrical Engineering, Mathematics and Computer Science,
        University of Twente, 7500 AE Enschede, The Netherlands
        {\tt\small h.kwakernaak@autsubmit.com}}%
\thanks{P. Misra is with the Department of Electrical Engineering, Wright State University,
        Dayton, OH 45435, USA
        {\tt\small pmisra@cs.wright.edu}}%
}


\begin{document}



\maketitle
\thispagestyle{empty}
\pagestyle{empty}


%%%%%%%%%%%%%%%%%%%%%%%%%%%%%%%%%%%%%%%%%%%%%%%%%%%%%%%%%%%%%%%%%%%%%%%%%%%%%%%%
\begin{abstract}

\end{abstract}


%%%%%%%%%%%%%%%%%%%%%%%%%%%%%%%%%%%%%%%%%%%%%%%%%%%%%%%%%%%%%%%%%%%%%%%%%%%%%%%%
\section{INTRODUCTION}
For a positive integer $N$, consider the linear systems
\begin{equation}\label{eq:linsys}
    x_{t+1} = Ax_{t} + \mathbf{B}\mathbf{u}_{t}
\end{equation}
where $t,i$ are nonegative integers, $m$ and $n$ are positive integers, $A \in \mathbb{R}^{n\times n}$, $\mathbf{B} := [B^{1},\cdots, B^{N}]$, $B^{i} \in \mathbb{R}^{n\times m}$, $\mathbf{u}_{t} = [u_{1,t}^{\transpose},\cdots,u_{N,t}^{\transpose}]^{\transpose}$, $x_t\in\mathbb{R}^n$, and $x_{1} =\bar{x}_{1}$ for some $\bar{x}_1 \in \mathbb{R}^{n}$, and $u_{t} \in \mathbb{R}^{m}$.
Define the following notations
\begin{align}
    &\{u_{i,t}\}_{i=1,t=1}^{N,T-1} := \{u_{1,1},\cdots,u_{N,1},\cdots, u_{1,T-1},\cdots,u_{N,T-1}\}\\
    \begin{split}
         &\{u_{i,t}\}_{i=1,t=1}^{N,\tau-1} \frown \{v_{i,t}\}_{i=1,t=\tau}^{N,T-1}:=\{u_{1,1},\cdots,u_{1,\tau-1},v_{1,\tau},\cdots,\\
    &\qquad v_{1,T-1},\cdots,u_{N,1},\cdots,u_{N,\tau-1},v_{N,\tau},\cdots,v_{N,T-1} \}
    \end{split}
   \\ 
   & R_{t}^{i} := 
   \begin{bmatrix}
       [R_{t}^{i}]_{11} & \cdots & [R_{t}^{i}]_{1N}\\
       \vdots & \vdots & \vdots\\
       [R_{t}^{i}]_{N1} & \cdots & [R_{t}^{i}]_{NN}
   \end{bmatrix}\\
    \label{eq:history}
    &\mathcal{H}_{T} := \{ \bar{x}_{1},\{Q_{t}^{i}\}_{i=1,t=1}^{N,T},\{R_{t}^{i}\}_{i=1,t=1}^{N,T-1}\}\\
    &g_{i,t}(x_{t}, \mathbf{u}_{t}) := \frac{1}{2}(x_{t}^{\mathsf{T}}Q_{t}^{i}x_{t} + 
    \mathbf{u}_{t}^{\transpose}R_{t}^{i}\mathbf{u}_{t}),\\
    &g_{i,T}(x) := \frac{1}{2} x^{\mathsf{T}}Q_{T}^{i}x
\end{align}
where for all $1 \leq t \leq T$ and $1\leq i,j\leq N$, $Q_{t}^{i}$ is symmetric, positive semi-definite, $R_{t}^{i,j}$ is positive definite. We again define
\begin{equation}
    J_{i,T}(\{x_{t}\}_{t=1}^{T},\{u_{i,t}\}_{i=1,t=1}^{N,T-1}) := \sum_{t=1}^{T-1} g_{i,t}(x_{t}, \mathbf{u}_{t}) + g_{i,T}(x_{T})
\end{equation}

Before we start the discussion of feedback Nash equilibrium, we use $\{u_{i,t}^{*}\}_{i=1,t=1}^{N,T-1}$ to denote the control sequence for feedback Nash equilibrium, and $\{u_{i,t}\}_{i=1,t=1}^{N,T-1}$ is sequence that consisted by arbitrary control vectors for each players. 

Furthermore, for $1 \leq \tau \leq T$, $\{x_{t}^{*\tau}\}_{t=1}^{T}$ is generated by the sequence of $\{u_{i,t}\}_{i=1,t=1}^{N,\tau-1} \frown \{u_{i,t}^{*}\}_{i=1,t=\tau}^{N,T}$, where $u_{i,t}$ can be an arbitrary vector from $\mathbb{R}^{m}$ for $1 \leq i \leq N$ and $1 \leq t \leq T$. Moreover, for $1 \leq \tau \leq T$ and $1 \leq i \leq N$, $\{x_{t}^{(i,K)}\}_{t=1}^{T}$ is generated by $\{u_{i,t}\}_{i=1,t=1}^{N,\tau-1} \frown \{u_{1,\tau}^{*},\dots, u_{i,\tau},\dots,u_{N,\tau}^{*}\} \frown \{u_{i,t}^{*}\}_{i=1,t=\tau+2}^{N,T}$. 

The sequence $\{ u_{i,t}^{*}\}_{i=1,t=1}^{N,T}$ satisfy that, for any fix $i$ that $1 \leq i \leq N$, such 
sequence satisfy that 
\begin{equation}\label{eq:nashIneq}
    \begin{split}
        \text{Level T}
        &\begin{cases}
            &J_{1,T}(\{x_{t}^{*T}\}_{t=1}^{T}, \{u_{i,t}\}_{i=1,t=1}^{N,T-2} \frown \{u_{1,T}^{*},u_{2,T}^{*},\dots,u_{N,T}^{*}\}) \\ & \leq J_{1,T}(\{x_{t}^{(1,T)}\}_{t=1}^{T}, \{u_{i,t}\}_{i=1,t=1}^{N,T-2} \frown \{u_{1,T},u_{2,T}^{*},\dots,u_{N,T}^{*}\})\\ \\
            &J_{2,T}(\{x_{t}^{*T}\}_{t=1}^{T}, \{u_{i,t}\}_{i=1,t=1}^{N,T-2} \frown \{u_{1,T}^{*},u_{2,T}^{*},\dots,u_{N,T}^{*}\}) \\ & \leq J_{2,T}(\{x_{t}^{(2,T)}\}_{t=1}^{T}, \{u_{i,t}\}_{i=1,t=1}^{N,T-2} \frown \{u_{1,T}^{*},u_{2,T},\dots,u_{N,T}^{*}\})\\
            & \qquad \qquad \vdots \\
            &J_{N,T}(\{x_{t}^{*T}\}_{t=1}^{T}, \{u_{i,t}\}_{i=1,t=1}^{N,T-2} \frown \{u_{1,T}^{*},u_{2,T}^{*},\dots,u_{N,T}^{*}\}) \\ & \leq J_{N,T}(\{x_{t}^{(N,T)}\}_{t=1}^{T}, \{u_{i,t}\}_{i=1,t=1}^{N,T-2} \frown \{u_{1,T}^{*},u_{2,T}^{*},\dots,u_{N,T}\})
        \end{cases}
    \\ &\qquad \qquad \qquad \vdots \\
    \text{Level 1}
        &\begin{cases}
            &J_{1,T}(\{x_{t}^{*1}\}_{t=1}^{T}, \{u_{1,1}^{*},u_{2,1}^{*},\dots,u_{N,1}^{*}\} \frown \{u_{i,t}^{*}\}_{i=1,t=2}^{N,T-1}) \\ & \leq J_{1,T}(\{x_{t}^{(1,1)}\}_{t=1}^{T}, \{u_{1,1},u_{2,1}^{*},\dots,u_{N,1}^{*}\} \frown \{u_{i,t}^{*}\}_{i=1,t=2}^{N,T-1})\\ \\
            &J_{2,T}(\{x_{t}^{*1}\}_{t=1}^{T}, \{u_{1,1}^{*},u_{2,1}^{*},\dots,u_{N,1}^{*}\} \frown \{u_{i,t}^{*}\}_{i=1,t=2}^{N,T-1}) \\ & \leq J_{2,T}(\{x_{t}^{(2,1)}\}_{t=1}^{T}, \{u_{1,1}^{*},u_{2,1},\dots,u_{N,1}^{*}\} \frown \{u_{i,t}^{*}\}_{i=1,t=2}^{N,T-1})\\
            & \qquad \qquad \vdots \\
            &J_{N,T}(\{x_{t}^{*1}\}_{t=1}^{T}, \{u_{1,1}^{*},u_{2,1}^{*},\dots,u_{N,1}^{*}\} \frown \{u_{i,t}^{*}\}_{i=1,t=2}^{N,T-1}) \\ & \leq J_{N,T}(\{x_{t}^{(N,1)}\}_{t=1}^{T}, \{u_{1,1}^{*},u_{2,1}^{*},\dots,u_{N,1}\} \frown \{u_{i,t}^{*}\}_{i=1,t=2}^{N,T-1})
        \end{cases}.
    \end{split}
\end{equation}
For convenient use later, we define the notion of $\text{DFLGame}$ as
\begin{equation}\label{eq:regret}
 (\{x_{t}^{*1}\}_{t=1}^{T}, \{u_{i,t}^{*}\}_{i=1,t=1}^{N,T}) := \text{DFLGame}(\mathcal{H}_{T},T).
\end{equation}
For any decisions $\usequence{N}{T-1}$ and the associated state sequence $\{x_{t}\}_{t=1}^{T}$, the dynamic social regret is defined as
\begin{equation}
    \begin{split}
        &\text{Regret}_{T}(\{\mathbf{u}_{t}\}_{t=1}^{T-1}) := \\
        &\sum_{i=1}^{N} J_{i,T}(\{x_{t}\}_{t=1}^{T},\{u_{i,t}\}_{i=1,t=1}^{N,T-1}) - J_{i,T}(\{x_{t}^{*}\}_{t=1}^{T},\{u_{i,t}^{*}\}_{i=1,t=1}^{N,T-1})
    \end{split}
\end{equation}


The key contributions of this paper are
\begin{itemize}
    \item The proposal of a method to solve a online LQ potential game;
    \item Development of the social-regret upperbound;
\end{itemize}


%%%%%%%%%%%%%%%%%%%%%%%%%%%%%%%%%%%%%%%%%%%%%%%%%%%%%%%%%%%%%%%%%%%%%%%%%%%%%%%%
\section{PROBLEM FORMULATION}
In this paper, we consider the following problem
\begin{problem}[Online Dynamic LQ Game]
     Consider the controllable system \eqref{eq:linsys}. Let the cost matrices in \eqref{eq:regret} satisfy Assumption \ref{assumption:cost} for any given $T \geq 1$ and $W < T-1$. At time $0 \leq t \leq T-W-1$, the available information to the decision maker is given by $\mathcal{H}_{t}$ as defined in \eqref{eq:history} and the current state $x_{t}$. It is desired to design a control policy $\pi(\cdot, \cdot)$ of the form \eqref{eq:policyForm} that yields a regret, as defined by \eqref{eq:regret}, that is independent of the bounds given in Assumption \ref{assumption:cost}.
\end{problem}

\section{APPROACH AND REGRET ANALYSIS}
In the following sections, we consider the special case of $N = 2$. 
\paragraph{Prediction. } 
Define
\begin{equation}
    [\bar{R}_{t}]_{pq} := [R_{t}^{p}]_{pq}
\end{equation}
for $1 \leq p,q \leq 2$. Given a preview window length $W$, define
\begin{equation}
\begin{split}
    \Pi_{\tau|t}(x_{\tau},\mathbf{u}_{\tau}) := 
    \begin{cases}
        \frac{1}{2} (x_{\tau}^{\transpose}Q_{\tau}x_{\tau} +  \mathbf{u}_{\tau}^{\transpose}\bar{R}_{\tau}\mathbf{u}_{\tau}) \text{\quad if $\tau \leq t+W$}\\
        \frac{1}{2} (x_{\tau}^{\transpose}Q_{t}x_{\tau} + \mathbf{u}_{\tau}^{\transpose}\bar{R}_{t+W}\mathbf{u}_{\tau}) \text{\quad if $t+W \leq \tau \leq T-1$}.
    \end{cases}
\end{split}
\end{equation}

At each time $t$, we plan an optimal trajectory starting from the initial state $\bar{x}_{1}$ using the known cost matrices up to time $t+W$ and setting all the future matrices to be equal to their known values for time $t+W$.

Specifically, at time $t$ where $0\leq t < T-W$, define $\Psi_{t}(\cdot,\cdot)$ as
\begin{align}
\Psi_{t}(\{\xi_{\tau}\}_{\tau=1}^{T},\{\boldsymbol{\upsilon}_{\tau}\}_{\tau=1}^{T-1}) &:= \sum_{\tau=1}^{T-1} \Pi_{\tau|t}(\xi_{\tau},\boldsymbol{\upsilon}_{\tau}) + \frac{1}{2} \xi_{T}^{\transpose}Q_{t+W}\xi_{T},
\end{align}
and 
\begin{equation}
    \Psi_{t}(\{\xi_{\tau}\}_{\tau=1}^{T},\{\boldsymbol{\upsilon}_{\tau}\}_{\tau=1}^{T-1}) = J_{t}(\{\xi_{\tau}\}_{\tau=1}^{T},\{\boldsymbol{\upsilon}_{\tau}\}_{\tau=1}^{T-1})
\end{equation}
for $T-W \leq t \leq T-1$.
Then, we find the predicted optimal control sequence for all $0\leq \tau\leq T-1$ by solving
\begin{equation}
\begin{split}
    (\{x_{\tau|t}\}_{\tau=1}^{T}, \{\mathbf{u}_{\tau|t}\}_{\tau=1}^{T-1}) \in &\arg\min_{\{\xi_{\tau}\}_{\tau=1}^{T},\{\upsilon_{\tau}\}_{\tau=1}^{T-1}} \Psi_{t}(\{\xi_{\tau}\}_{\tau=1}^{T},\{\upsilon_{\tau}\}_{\tau=1}^{T-1})\\
    &\text{subject to}\quad \xi_{\tau+1} = A\xi_{\tau} + \mathbf{B}\mathbf{\upsilon_{\tau}}, \xi_{1} = \bar{\xi}_{1}
\end{split}
\end{equation}

\paragraph{Prediction Tracking. } We propose the following feedback control policy
\begin{equation}
    \pi(x_{t},\mathbf{H}_{t}) := \mathbf{K}(x_{t}-x_{t|t}) + \mathbf{u}_{t|t},
\end{equation}
where $\mathbf{K}\in \mathbb{R}^{m\times n}$ is a control matrix such that $\rho(A+\mathbf{B}\mathbf{K}) < 1$, and $\rho(\cdot)$ denotes the matrix spectral radius.


\subsection{Regret Analysis}

\begin{assumption}
    For any given $p,q(p={1,2}\text{ and }q={1,2})$, there exists positive definite matrices $Q_{min}, Q_{max}, R_{min}, R_{max}$ that
    \begin{equation}
        Q_{min} \preceq Q_{t}^{p} \preceq Q_{max}
    \end{equation}
    for $1\leq t \leq T$, and
    \begin{equation}
        R_{min} \preceq [R_{t}^{p}]_{p,q} \preceq R_{max}
    \end{equation}
    for $1 \leq t \leq T-1$.
\end{assumption}

\begin{definition}
    The dynamic game is referred to as a feedback potential difference game(FPDG) if there exist an optimal control problem (OCP) such that the solution of the OCP, in feedback form, provides a feedback Nash equlibrium for the linear quadratic dynamic game. 
\end{definition}

\begin{lemma}
    If the cost matrices $Q_{t}$, $R_{t}^{i}$ satisfy the following,
    \begin{enumerate}
        \item for time instant $t$ from $T-1$ to 1, $R_{t}^{1},R_{t}^{2},Q_{t}$ satisfy
        \begin{equation}\label{eq:costFPDG1}
            [R_{t}^{1}]_{12} + B^{1\transpose}P_{t+1}^{1}B^{2} = ([R_{t}^{2}]_{21} + B^{2\transpose}P_{t+1}^{2}B^{1})^{\transpose}.
        \end{equation}
        Define
        \begin{equation}
        [\Theta_{t}]_{ij} := [R_{t}]^{i}_{ij} + B^{i\transpose}P_{t+1}^{i}B^{j},
        \end{equation}
        we require
        \begin{equation}
            \Theta_{t} \succ 0,
        \end{equation}
        \begin{equation*}
            \mathbf{B}^{\transpose}(P_{t}^{1}-P_{t}^{2})A=0.
        \end{equation*}
        \item at time instant 1: $(R_{1}^{1},R_{1}^{2})$ satisfy
        \begin{equation}
            [R_{1}^{1}]_{12} + B^{1\transpose}P_{1}^{1}B^{2} = ([R_{1}^{2}]_{21} + B^{2\transpose}P_{1}^{2}B^{1})^{\transpose},
        \end{equation}
        \begin{equation}\label{eq:costFPDG2}
            \Theta_{1} \succ 0.
        \end{equation}
    \end{enumerate}
    Then, the dynamic game is an feedback potential difference game(FPDG).
\end{lemma}
The above lemma is a special case of \cite{}[Theorem 6] when $Q_{t}^{1}=Q_{t}^{2}$ for $t(1\leq t \leq T)$.

\begin{corollary}
    Suppose the cost matrices $\{Q_{t}\}_{t=1}^{T}$ and $\{R_{t}^{i}\}_{t=1,i=1}^{T-1,2}$ satisfy the conditions described from \eqref{eq:costFPDG1} to \eqref{eq:costFPDG2}.
    For a given preview horizon $W(0\leq W \leq T-1)$, define
    \begin{equation}
        Q_{\tau|t}:= 
        \begin{cases}
            Q_{\tau} \text{ if $1\leq \tau \leq t+W$}\\
            Q_{t+w} \text{ if $t+W < \tau \leq T$}
        \end{cases}
    \end{equation}
    and
    \begin{equation}
        R_{\tau|t}^{i}:= 
        \begin{cases}
            R_{\tau}^{i} \text{ if $1\leq \tau \leq t+W$}\\
            R_{t+w}^{i} \text{ if $t+W < \tau \leq T$}
        \end{cases}
    \end{equation}
    for $i = {1,2}$.
    If the cost matrices further satisfy
    \begin{equation}
        \mathbf{K}_{\tau|t}^{\transpose}(R_{\tau|t}^{1}-R_{\tau|t}^{2})\mathbf{K}_{\tau|t} = 0.
    \end{equation}
    Then all the cost matrices $\bar{Q}_{\tau|t}$ and $\bar{R}_{\tau|t}$ can be determined by $Q_{\tau|t}$ and $R_{\tau|t}^{i}$ for $1\leq t \leq T$ and $i = \{1,2\}$.
\end{corollary}

\begin{lemma}
    
\end{lemma}



\begin{lemma}
    The regret can be written as
    \begin{equation}
        \text{Regret}_{T}(\{\mathbf{u}_{t}\}_{t=1}^{T-1}) = \frac{1}{2}\sum_{t=1}^{T-1} x_{t}^{\transpose}Q_{t}x_{t} + \mathbf{u}_{t}^{\transpose}\tilde{\mathbf{R}}_{t}\mathbf{u}_{t} - x_{t}^{*\transpose}Q_{t}x_{t}^{*} - \mathbf{u}_{t}^{*\transpose}\tilde{\mathbf{R}}_{t}\mathbf{u}_{t}^{*},
    \end{equation}
    where 
    \begin{align*}
        \tilde{\mathbf{R}}_{t} = 
        \begin{pmatrix}
            \sum_{j=1}^{N} R_{t}^{j,1} & 0 & \cdots & 0\\
            0 & \sum_{j=1}^{N} R_{t}^{j,2} & \cdots & 0\\
            \vdots& \vdots & \vdots & \vdots\\
            0 & \cdots & 0 & \sum_{j=1}^{N} R_{t}^{j,N}
        \end{pmatrix}
    \end{align*}
\end{lemma}

\begin{lemma}
    Suppose matrices $Q_{t}$ and $\tilde{\mathbf{R}}_{t}$ are lower and upper bounded by $Q_{min},Q_{max},\contTilde{R}_{min},\contTilde{R}_{max}$, the regret can be upperbounded by
    \begin{equation}
        \text{Regret}_{T}(\{\mathbf{u}_{t}\}_{t=1}^{T-1}) \leq \frac{1}{2}\sum_{t=1}^{T-1} \|Q_{max}\|\|x_{t}-x_{t}^{*}\|^2 + \|\contTilde{R}_{max}\|\|\mathbf{u}_{t}-\mathbf{u}_{t}^{*}\|^2
    \end{equation}
\end{lemma}




%%%%%%%%%%%%%%%%%%%%%%%%%%%%%%%%%%%%%%%%%%%%%%%%%%%%%%%%%%%%%%%%%%%%%%%%%%%%%%%%
\section{NUMERICAL SIMULATIONS}

\section{CONCLUSIONS AND FUTURE WORKS}

%%%%%%%%%%%%%%%%%%%%%%%%%%%%%%%%%%%%%%%%%%%%%%%%%%%%%%%%%%%%%%%%%%%%%%%%%%%%%%%%
\section{ACKNOWLEDGMENTS}

The authors gratefully acknowledge the contribution of National Research Organization and reviewers' comments.


%%%%%%%%%%%%%%%%%%%%%%%%%%%%%%%%%%%%%%%%%%%%%%%%%%%%%%%%%%%%%%%%%%%%%%%%%%%%%%%%

References are important to the reader; therefore, each citation must be complete and correct. If at all possible, references should be commonly available publications.

\begin{thebibliography}{99}

\bibitem{c1}
J.G.F. Francis, The QR Transformation I, {\it Comput. J.}, vol. 4, 1961, pp 265-271.

\bibitem{c2}
H. Kwakernaak and R. Sivan, {\it Modern Signals and Systems}, Prentice Hall, Englewood Cliffs, NJ; 1991.

\bibitem{c3}
D. Boley and R. Maier, "A Parallel QR Algorithm for the Non-Symmetric Eigenvalue Algorithm", {\it in Third SIAM Conference on Applied Linear Algebra}, Madison, WI, 1988, pp. A20.

\end{thebibliography}

\end{document}
